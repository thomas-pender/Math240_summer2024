\documentclass[a4paper,11pt]{article}

\usepackage[utf8]{inputenc}
\usepackage[english]{babel}
\usepackage{amssymb, amsmath, amsthm, mathrsfs}
\usepackage[left=1.0in,right=1.0in,top=1.0in,bottom=1.0in]{geometry}
\usepackage[inline,shortlabels]{enumitem}
\usepackage{times}
\usepackage{xcolor}

\newcommand{\R}{\mathbf{R}}

\begin{document}

\noindent{\bfseries Question.} Let $m$ and $n$ be positive integers such that $m
\geqq n$, and let $a_1,\dots,a_m,c \in \R$ with $c \neq 0$ and
$\text{span}\{a_1,\dots,\,a_m\} = \R^n$. Prove there exists an $i \in
\{1,\dots,\,m\}$ for which
\[
  \text{span}\{a_1,\dots,\,a_{i-1},\,c,\,a_{i+1},\dots,\,a_m\} = \R^n.
\] \\

\noindent{\bfseries Solution.} We will prove the result by contradiction. The
logical form of the proposition is
\begin{align*}
  (\forall m \geqq n)(\forall a_1,\dots,\,a_m,\,c \in \R^n)
  &\Big[
  \text{span}\{a_1,\dots,a_m\}=\R^n \Rightarrow
  \Big(
  c \neq 0 \Rightarrow \\
  &(\exists i \in \{1,\dots,\,m\})
  (\text{span}\{a_1,\dots,\,a_{i-1},\,c,\,a_{i+1},\dots,\,a_m\} = \R^n)
  \Big)
  \Big]
\end{align*}
To prove the result by contradiction, we assume the logical negation of the
statement. This is given by
\begin{align*}
  (\exists m \geqq n)(\exists a_1,\dots,\,a_m,\,c \in \R^n)
  &\Big[
  \text{span}\{a_1,\dots,a_m\}=\R^n \text{ \& } c \neq 0 \text{ \& } \\
  &(\forall i \in \{1,\dots,\,m\})
  (\text{span}\{a_1,\dots,\,a_{i-1},\,c,\,a_{i+1},\dots,\,a_m\} \neq \R^n)
  \Big]
\end{align*}

So, to prove the result by contradiction, we assume there are positive integers
$m \geqq n$ such that there exists $a_1,\dots,\,a_m,\,c \in \R^n$ for which
$\text{span}\{a_1,\dots,\,a_m\} = \R^n$ and $c \neq 0$ and, for every index $i
\in \{1,\dots,\,m\}$,
\[
  \text{span}\{a_1,\dots,\,a_{i-1},\,c,\,a_{i+1},\dots,\,a_m\} \neq \R^n.
\]

Choose an arbitrary $i \in \{1,\dots,\,m\}$. By assumption, there is some $v \in
\R^n$ such that $v \notin
\text{span}\{a_1,\dots,\,a_{i-1},\,c,\,a_{i+1},\dots,\,a_m\}$. Since
$\text{span}\{a_1,\dots,\,a_m\} = \R^n$, there are scalars
$\alpha_1,\dots,\,\alpha_m \in \R$ such that
\[
  v = \alpha_1a_1 + \cdots + \alpha_ia_i + \cdots + \alpha_ma_m.
\]
If $\alpha_i = 0$, then
\[
  v \in \text{span}\{a_1,\dots,\,a_{i-1},\,a_{i+1},\dots,\,a_m\}
    \subseteqq
    \text{span}\{a_1,\dots,\,a_{i-1},\,c,\,a_{i+1},\dots,\,a_m\}
\]
which contradicts our assumptions about $v$; so, $\alpha_i \neq 0$.

Since $c \in \R^n=\text{span}\{a_1,\dots,\,a_m\}$, there are scalars
$\beta_1,\dots,\,\beta_m \in \R$ such that
\[
  c = \beta_1a_1 + \cdots + \beta_ia_i + \cdots + \beta_ma_m.
\]
Assume that $\beta_i \neq 0$. Then there is a unique $x \in \R$ such that $x
\neq 0$ and $\alpha_i = x\beta_i$. But then
\begin{align*}
  &(\alpha_1-x\beta_1)a_1 + \cdots + (\alpha_{i-1}-x\beta_{i-1})a_{i-1} +
  xc + (\alpha_{i+1}-x\beta_{i+1})a_{i+1} + \dots + (\alpha_m-x\beta_m)a_m \\
  &= \alpha_1a_1 + \cdots + \alpha_{i-1}a_{i-1} + x\beta_ia_i +
    \alpha_{i+1}a_{i+1} + \cdots + \alpha_ma_m \\
  &= \alpha_1a_1 + \cdots + \alpha_{i-1}a_{i-1} + \alpha_ia_i +
    \alpha_{i+1}a_{i+1} + \cdots + \alpha_ma_m \\
  &= v
\end{align*}
which expresses $v$ as a linear combination of
$a_1,\dots,\,a_{i-1},\,c,\,a_{i+1}\,\dots,\,a_m$, a contradiction. Thus,
$\beta_i = 0$. Since $i$ was arbitrary, $\beta_i = 0$ for every $i \in
\{1,\dots,\,m\}$. But this means that $c=0$, contrary to our assumption.
Therefore, our assumption that the proposition was false is incorrect, that is,
it must be true.

\end{document}