\documentclass[a4paper,11pt]{article}

\usepackage[utf8]{inputenc}
\usepackage[english]{babel}
\usepackage{amssymb, amsmath, amsthm, mathrsfs}
\usepackage[left=1.0in,right=1.0in,top=1.0in,bottom=1.0in]{geometry}
\usepackage[inline,shortlabels]{enumitem}
\usepackage{times}
\usepackage{xcolor}

\newcommand{\R}{\mathbf{R}}
\newcommand{\BB}[1]{\textcolor{blue}{#1}}

\begin{document}

\begin{center}
  {\Large\bfseries Math 240 Tutorial \\ Solutions}
\end{center}
\begin{center}
  {\bfseries May 23}
\end{center}

\noindent{\bfseries Question 1.} For each part, explain whether or not the stated
matrix--vector multiplication can be carried out. If it can, do the multiplication.
\begin{enumerate}[(a)]
\item
  \[
    \begin{pmatrix}
      1&1\\1&-1
    \end{pmatrix}
    \begin{pmatrix}
      1\\1\\0
    \end{pmatrix}.
  \]

  \BB{ The matrix and the vector are not conformable; to be specific, the number
of columns of the matrix does not equal the number of entries in the column. So,
the multiplication cannot be carried out. \\ }
  
\item
  \[
    \begin{pmatrix}
      x & 1 & 1 \\
      1 & x & 1 \\
      1 & 1 & x
    \end{pmatrix}
    \begin{pmatrix}
      1\\0\\0
    \end{pmatrix}.
  \] \\

  \BB{ Here the matrix and the vector are conformable. Their product is given by
    \[
      \begin{pmatrix}
        x\\1\\1
      \end{pmatrix}.
    \]
  }
  
\end{enumerate}

\noindent{\bfseries Question 2.} Write the following linear system first as a
vector equation and then as a matrix equation
\begin{align*}
  u+2v-w-2x+3y &= b_1, \\
  x-y+2z &= b_2, \\
  2u+4v-2w-4x+7y-4z &= b_3, \\
  -x+y-2z &= b_4, \\
  3u+6v-3w-6x+7y+8z &= b_5,
\end{align*}
where $b_1,\,b_2,\,b_3,\,b_4,\,b_5 \in \R$. \\

\BB{ The vector equation is given by
  \[
    u\begin{pmatrix}1\\0\\2\\0\\3\end{pmatrix}+
    v\begin{pmatrix}2\\0\\4\\0\\6\end{pmatrix}+
    w\begin{pmatrix}-1\\0\\-2\\0\\-3\end{pmatrix}+
    x\begin{pmatrix}-2\\1\\-4\\-1\\-6\end{pmatrix}+
    y\begin{pmatrix}3\\-1\\7\\1\\7\end{pmatrix}+
    z\begin{pmatrix}0\\2\\-4\\-2\\8\end{pmatrix}=
    \begin{pmatrix}b_1\\b_2\\b_3\\b_4\\b_5\end{pmatrix}.
  \]
}

\BB{ The matrix equation is given by
  \[
    \begin{pmatrix}
      1 & 2 & -1 & -2 & 3 & 0 \\
      0 & 0 & 0 & 1 & -1 & 2 & \\
      2 & 4 & -2 & -4 & 7 & -4 \\
      0 & 0 & 0 & -1 & 1 & -2 \\
      3 & 6 & -3 & -6 & 7 & 8 
    \end{pmatrix}
    \begin{pmatrix}u\\v\\w\\x\\y\\z\end{pmatrix}
    =
    \begin{pmatrix}b_1\\b_2\\b_3\\b_4\\b_5\end{pmatrix}.
  \]
}

\noindent{\bfseries Question 3.} For each of the following lists of row vectors in
$\R^3$, determine whether the first vector can be expressed as a linear
combination of the other two vectors.
\begin{enumerate}[(a)]
\item $(-2,0,3)$, $(1,3,0)$, $(2,4,-1)$. \\

  \BB{We have
    \[
      \left(
        \begin{array}{rr|r}
          1 & 2 & -2 \\
          3 & 4 & 0 \\
          0 & -1 & 3
        \end{array}
      \right) 
      \sim
      \left(
        \begin{array}{rr|r}
          1 & 0 & 4 \\
          0 & 1 & -3 \\
          0 & 0 & 0
        \end{array}
      \right).
    \]
    Therefore, $(-2,0,3) \in \text{span}\{(1,3,0),\,(2,4,-1)\}$. \\
  }

\item $(1,2,-3)$, $(-3,2,1)$, $(2,-1,-1)$. \\

  \BB{We have
    \[
      \left(
        \begin{array}{rr|r}
          -3 & 2 & 1 \\
          2 & -1 & 2 \\
          1 & -1 & -3
        \end{array}
      \right) 
      \sim
      \left(
        \begin{array}{rr|r}
          1 & 0 & 5 \\
          0 & 1 & 8 \\
          0 & 0 & 0
        \end{array}
      \right).
    \]
    Therefore, $(1,2,-3) \in \text{span}\{(-3,2,1),\,(2,-1,-1)\}$. \\
  }
  
\item $(3,4,1)$, $(1,-2,1)$, $(-2,-1,1)$. \\

  \BB{We have
    \[
      \left(
        \begin{array}{rr|r}
          1 & -2 & 3 \\
          -2 & -1 & 4 \\
          1 & 1 & 1
        \end{array}
      \right)
      \sim
      \left(
        \begin{array}{rr|r}
          1 & 0 & 0 \\
          0 & 1 & 0 \\
          0 & 0 & 1
        \end{array}
      \right).
    \]
    Therefore, $(1,2,-3) \notin \text{span}\{(-3,2,1),\,(2,-1,-1)\}$. \\
  }

\item $(2,-1,0)$, $(1,2,-3)$, $(1,-3,2)$. \\

  \BB{We have
    \[
      \left(
        \begin{array}{rr|r}
          1 & 1 & 2 \\
          2 & -3 & -1 \\
          -3 & 2 & 0
        \end{array}
      \right) 
      \sim
      \left(
        \begin{array}{rr|r}
          1 & 0 & 0 \\
          0 & 1 & 0 \\
          0 & 0 & 1
        \end{array}
      \right).
    \]
    Therefore, $(2,-1,0) \notin \text{span}\{(1,2,-3),\,(1,-3,2)\}$. \\
  }
  
\item $(5,1,-5)$, $(1,-2,-3)$, $(-2,3,-4)$. \\

  \BB{We have
    \[
      \left(
        \begin{array}{rr|r}
          1 & -2 & 5 \\
          -2 & 3 & 1 \\
          -3 & -4 & -5
        \end{array}
      \right) 
      \sim
      \left(
        \begin{array}{rr|r}
          1 & 0 & 0 \\
          0 & 1 & 0 \\
          0 & 0 & 1
        \end{array}
      \right).
    \]
    Therefore, $(5,1,-5) \notin \text{span}\{(1,-2,-3),\,(-2,3,-4)\}$. \\
  }
  
\item $(-2,2,2)$, $(1,2,-1)$, $(-3,-3,3)$. \\

  \BB{We have
    \[
      \left(
        \begin{array}{rr|r}
          1 & -3 & -2 \\
          2 & -3 & 2 \\
          -1 & 3 & 2
        \end{array}
      \right) 
      \sim
      \left(
        \begin{array}{rr|r}
          1 & 0 & 4 \\
          0 & 1 & 2 \\
          0 & 0 & 0
        \end{array}
      \right).
    \]
    Therefore, $(-2,2,2) \in \text{span}\{(1,2,-1),\,(-3,-3,3)\}$. \\
  }
\end{enumerate}

\noindent{\bfseries Question 4.} Consider the following three vectors in $\R^3$
\[
  \vec u_1 = \begin{pmatrix}1\\1\\0\end{pmatrix},\quad
  \vec u_2 = \begin{pmatrix}0\\1\\1\end{pmatrix},\quad
  \vec u_3 = \begin{pmatrix}1\\0\\1\end{pmatrix}.
\]
Show that $\R^3=\text{span}\{\vec u_1,\,\vec u_2,\,\vec u_3\}$. \\

\BB{We will prove this result in two ways. First, observing that
  \[
    \left(
      \begin{array}{rrr}
        1 & 0 & 1 \\
        1 & 1 & 0 \\
        0 & 1 & 1
      \end{array}
    \right)
    \sim
    \left(
      \begin{array}{rrr}
        1 & 0 & 0 \\
        0 & 1 & 0 \\
        0 & 0 & 1
      \end{array}
    \right)
  \]
  the result follows. If, however, we required more information in how a vector
  of $R^3$ is decomposable as a linear combination of $\vec u_1,\,\vec u_2,\,\vec
  u_3$, we could note the following
  \[
    \left(
      \begin{array}{rrrr}
        1 & 0 & 1 & a_1 \\
        1 & 1 & 0 & a_2 \\
        0 & 1 & 1 & a_3
      \end{array}
    \right) 
    \sim
    \left(
      \begin{array}{rrrr}
        1 & 0 & 0 & \frac{1}{2}a_1+\frac{1}{2}a_2-\frac{1}{2}a_3 \\
        0 & 1 & 0 & -\frac{1}{2}a_1+\frac{1}{2}a_2+\frac{1}{2}a_3 \\
        0 & 0 & 1 & \frac{1}{2}a_1-\frac{1}{2}a_2+\frac{1}{2}a_3
      \end{array}
    \right).
  \]
  This means that if
  $
  \left(
    \begin{smallmatrix}
      a_1\\a_2\\a_3
    \end{smallmatrix}
  \right)
  \in \R^3,
  $
  then
  \[
    \left( \frac{1}{2}a_1+\frac{1}{2}a_2-\frac{1}{2}a_3 \right)\vec u_1-
    \left( \frac{1}{2}a_1-\frac{1}{2}a_2-\frac{1}{2}a_3 \right)\vec u_2+
    \left( \frac{1}{2}a_1-\frac{1}{2}a_2+\frac{1}{2}a_3 \right)\vec u_3=
    \begin{pmatrix}a_1\\a_2\\a_3\end{pmatrix}.
  \]
}

\noindent{\bfseries Question 5.} Show that
$\text{span}\{\vec u,\,\vec v,\,\vec w\}=\text{span}\{\vec u,\,\vec v+\vec
w,\,\vec v-\vec w\}$. \\

\BB{Since each of $u,\,v+w,\,v-w$ are in $\text{span}\{u,\,v,\,w\}$, it follows
  that $\text{span}\{u,\,v+w,\,v-w\} \subseteqq \text{span}\{u,\,v,\,w\}$.
  Conversely, $\frac{1}{2}\Big( (v+w)+(v-w) \Big)=v$ and $\frac{1}{2}\Big(
  (v+w)-(v-w) \Big)=w$ so that $\text{span}\{u,\,v,\,w\} \subseteqq
  \text{span}\{u,\,v+w,\,v-w\}$. These two containments show the desired
  equality. \\}

\noindent{\bfseries Question 6.} Consider the following four vectors in $\R^4$
given by
\[
  \vec v_1 = \begin{pmatrix}+1 \\ -1 \\ -1 \\ -1\end{pmatrix},  \quad
  \vec v_2 = \begin{pmatrix}-1 \\ +1 \\ -1 \\ -1\end{pmatrix}, \quad
  \vec v_3 = \begin{pmatrix}-1 \\ -1 \\ +1 \\ -1\end{pmatrix}, \quad
  \vec v_4 = \begin{pmatrix}-1 \\ -1 \\ -1 \\ +1\end{pmatrix}.
\]
\begin{enumerate}[(a)]
\item\label{Q1: span} Show whether $\vec v_1 \in \text{span}\{\vec v_2,\,\vec
  v_3,\,\vec v_4\}$ 
  or not by solving the corresponding system of linear equations. \\

  \BB{Observe
    \[
      \left(
        \begin{array}{rrr|r}
          -1 & -1 & -1 & 1 \\
          1 & -1 & -1 & -1 \\
          -1 & 1 & -1 & -1 \\
          -1 & -1 & 1 & -1
        \end{array}
      \right) 
      \sim
      \left(
        \begin{array}{rrr|r}
          1 & 0 & 0 & 0 \\
          0 & 1 & 0 & 0 \\
          0 & 0 & 1 & 0 \\
          0 & 0 & 0 & 1
        \end{array}
      \right)
    \] so that $\vec v_1 \notin \text{span}\{\vec v_2,\,\vec v_3,\,\vec v_4\}$. \\
  }
  
\item\label{Q1: lin indep} Let $a_1,\,a_2,\,a_3,\,a_4 \in \R$. Under what
  conditions on $a_1,\,a_2,\,a_3,\,a_4$ is $a_1\vec v_1+a_2\vec v_2+a_3\vec
  v_3+a_4\vec v_4=\vec{0}$ true? \\

  \BB{We have
    \[
      \left(
        \begin{array}{rrrr}
          1 & -1 & -1 & -1 \\
          -1 & 1 & -1 & -1 \\
          -1 & -1 & 1 & -1 \\
          -1 & -1 & -1 & 1
        \end{array}
      \right) 
      \sim
      \left(
        \begin{array}{rrrr}
          1 & 0 & 0 & 0 \\
          0 & 1 & 0 & 0 \\
          0 & 0 & 1 & 0 \\
          0 & 0 & 0 & 1
        \end{array}
      \right)
    \]
    so that only the solution $a_1=a_2=a_3=a_4=0$ exists. \\
  }
  
\item How can we use part \ref{Q1: lin indep} to provide a second proof of part
  \ref{Q1: span}? Can you generalize to answer the following question: Is $\vec
  v_i \in \text{span}\{\vec v_j,\,\vec v_k,\,\vec v_l\}$ for $i,\,j,\,k,\,l$
  distinct? \\

  \BB{We show only the second question (the first being a special case of the
    second). We will use proof by contradiction. Assume to the contrary that
    there exists $a_j,\,a_k,\,a_l \in \R$ not all zero such that $\vec v_i
    = a_j\vec v_j + a_k\vec v_k + a_l\vec v_l$. But then $\vec v_i - a_j\vec v_j
    - a_k\vec v_k - a_l\vec v_l = \vec 0$. From our answer to part \ref{Q1: lin
      indep}, it follows that $a_j=a_k=a_l=0$ and $1=0$. But $1=0$ is a
    contradiction. Therefore, our original assumption that $\vec v_i \in
    \text{span}\{\vec v_j,\,\vec v_k,\,\vec v_l\}$ is incorrect. That is, we
    must have $\vec v_i \notin \text{span}\{\vec v_j,\,\vec v_k,\,\vec v_l\}$ \\}
\end{enumerate}

\noindent{\bfseries Question 7.} Let $\vec v_1,\,\vec v_2,\dots,\,\vec v_n \in
\R^n$ be such that if $a_1\vec v_1 + a_2\vec v_2 + \cdots + a_n\vec v_n=\vec 0$
then $a_1=a_2=\cdots=a_n=0$. Show this implies that every vector in
$\text{span}\{\vec v_1,\,\vec v_2,\dots,\,\vec v_n\}$ can be written {\it
  uniquely} as a linear combination of $\vec v_1,\,\vec v_2,\dots,\,\vec v_n$. \\

\BB{Let $\vec u \in \text{span}\{\vec v_1,\,\vec v_2,\dots,\,\vec v_n\}$. Then
there are scalars $a_1,\,a_2,\dots,\,a_n$ such that $\vec u = a_1\vec v_1 +
a_2\vec v_2 + \cdots + a_n\vec v_n$. Assume there is another collection
$b_1,\,b_2,\dots,\,b_n$ of scalars such that $\vec u = b_1\vec v_1 + b_2\vec v_2
+ \cdots + b_n\vec v_n$. Then
\[
  \vec 0 = \vec u - \vec u =
  (a_1-b_1)\vec v_1 + (a_2-b_2)\vec v_2 + \cdots + (a_n-b_n)\vec v_n.
\]
By assumption,
\[
  a_1-b_1 = a_2-b_2 = \cdots = a_n-b_n = 0
\]
so that
\[
  a_1 = b_1, \quad a_2 = b_2,\, \dots\quad, a_n = b_n
\] as desired. \\}

\noindent{\bfseries Question 8.} Let $V_1$ and $V_2$ be two subsets of $\R^n$, and
define $V_1+V_2=\{\vec v_1 + \vec v_2 : \vec v_1 \in V_1 \text{ and } \vec v_2
\in V_2 \}$. Show
\begin{enumerate*}[(a)]
\item $\text{span}(V_1 \cup V_2) = \text{span}(V_1) + \text{span}(V_2)$, and
\item $\text{span}(V_1 \cap V_2) \subseteqq \text{span}(V_1) \cap
  \text{span}(V_2)$.
\end{enumerate*}
Further, give an example of subsets $V_1$ and $V_2$ of $\R^n$, for some $n$, for
which $\text{span}(V_1 \cap V_2) \subsetneqq \text{span}(V_1) \cap
\text{span}(V_2)$. \\

\BB{
  Throughout, let $V_1 \cap V_2 = \{w_1,\,w_2,\dots,\,w_p\}$,
  $V_1=\{\vec v_1,\,\vec v_2,\dots,\,\vec v_{n-p},\,\vec w_1,\,\vec
  w_2,\dots,\,\vec w_p\}$, $V_2=\{\vec u_1,\,\vec u_2,\dots,\,\vec
  u_{m-p},\,\vec w_1,\,\vec w_2,\dots,\,\vec w_p\}$.
  \begin{enumerate}[(a)]
  \item Let $\vec x \in \text{span}(V_1 \cup V_2)$. Then there exists scalars
    $a_1,\,a_2,\dots,\,a_{n-p} \in \R$ and $b_1,\,b_2,\dots,\,b_{m-p} \in \R$
    and $c_1,\,c_2,\dots,\,c_p \in \R$ such that 
    \[
      \vec x = \sum_{i=1}^{n-p}a_i\vec v_i + \sum_{j=1}^{m-p}b_j\vec u_j +
      \sum_{k=1}^pc_k\vec w_k.
    \]
    But
    \[
      \sum_{i=1}^{n-p}a_i\vec v_i + \sum_{k=1}^pc_k\vec w_k \in \text{span}(V_1)
    \]
    and
    \[
      \sum_{j=1}^{m-p}b_j\vec u_j \in \text{span}(V_2).
    \]
    Therefore, $\vec x \in \text{span}(V_1)+\text{span}(V_2)$, and
    $\text{span}(V_1 \cup V_2) \subseteqq \text{span}(V_1)+\text{span}(V_2)$.
    Conversely, let $\vec y \in \text{span}(V_1)+\text{span}(V_2)$. Then there
    exists scalars $\alpha_1,\,\alpha_2,\dots,\,\alpha_n \in \R$ and
    $\beta_1,\,\beta_2,\dots,\,\beta_m \in \R$ such that
    \[
      y = \sum_{i=1}^{n-p}\alpha_i\vec v_i +
      \sum_{j=1}^p(\alpha_j+\beta_j)\vec w_j +
      \sum_{k=1}^{m-p}\beta_k\vec u_k.
    \]
    Since $V_1 \cup V_2=\{\vec v_1,\,\dots,\,\vec v_n,\,\vec u_1,\dots,\,\vec
    u_m,\,\vec w_1,\dots,\,\vec w_p\}$, we have that $\vec y \in \text{span}(V_1
    \cup V_2)$, and $\text{span}(V_1 \cup V_2) \supseteqq
    \text{span}(V_1)+\text{span}(V_2)$.
  \item Let $\vec x \in \text{span}(V_1 \cap V_2)$. Then there exists scalars
    $a_1,\,a_2,\dots,\,a_p$ such that
    \[
      \vec x = a_1\vec w_1 + a_2\vec w_2 + \cdots + a_p\vec w_p.
    \]
    Since $V_1 \cap V_2 \subseteqq V_1$ and $V_1 \cap V_2 \subseteqq V_2$, we
    see at once that $\vec x$ is in both $\text{span}(V_1)$ and
    $\text{span}(V_2)$, that is, $\vec x \in
    \text{span}(V_1)\cap\text{span}(V_2)$, as desired. Any number of counter
    examples can be found to show that $\text{span}(V_1 \cap V_2) =
    \text{span}(V_1) \cap \text{span}(V_2)$ is not true in general. It isn't
    difficult to see that $V_1=\left\{
      \left( \begin{smallmatrix}1\\0\end{smallmatrix}
      \right),\,\left( \begin{smallmatrix}0\\1\end{smallmatrix} \right)
    \right\}$  and $V_2=\left\{
      \left( \begin{smallmatrix}1\\1\end{smallmatrix} \right) \right\}$ give
    a counter example. Since then $\text{span}(V_1 \cap V_2)=\{\vec 0\}$ and
    $\text{span}(V_1)\cap\text{span}(V_2)=\text{span}(V_2) \neq \{\vec 0\}$.
  \end{enumerate}
}

\end{document}