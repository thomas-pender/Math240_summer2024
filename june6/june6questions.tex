\documentclass[a4paper,11pt]{article}

\usepackage[utf8]{inputenc}
\usepackage[english]{babel}
\usepackage{amssymb, amsmath, amsthm, mathrsfs}
\usepackage[left=1.0in,right=1.0in,top=1.0in,bottom=1.0in]{geometry}
\usepackage[inline,shortlabels]{enumitem}
\usepackage{times}
\usepackage{xcolor}

\newcommand{\R}{\mathbf{R}}

\begin{document}

\begin{center}
  {\Large\bfseries Math 240 Tutorial \\ Questions}
\end{center}
\begin{center}
  {\bfseries June 6}
\end{center}

\subsection*{Systems of Linear Equations and Row Reduction}

\noindent{\bfseries Question 1.} Place the following augmented matrices into
an echelon form. Does the corresponding system of linear equations admit any
solutions?
\begin{enumerate}[(a)]
\item
  \[
    \left(
      \begin{array}{cccc|c}
        4 & 8 & 12 & 4 & 7 \\
        2 & 5 & 6 & 6 & 11 \\
        0 & 5 & 1 & 26 & 13 \\
        0 & 5 & 0 & 21 & 17
      \end{array}
    \right).
  \]
\item
  \[
    \left(
      \begin{array}{cccc|c}
        4 & 8 & 12 & 4 & 0 \\
        2 & 5 & 6 & 6 & 0 \\
        0 & 5 & 1 & 25 & 0 \\
        0 & 5 & 0 & 20 & 0       
      \end{array}
    \right).
  \]
\item
  \[
    \left(
      \begin{array}{cccc|c}
        4 & 8 & 12 & 4 & 7 \\
        2 & 5 & 6 & 6 & 11 \\
        0 & 5 & 1 & 25 & 13 \\
        0 & 5 & 0 & 20 & 17       
      \end{array}
    \right).
  \]\\
\end{enumerate}

\noindent{\bfseries Question 2.} Find the values of $k$ for which the system of
equations
\begin{align*}
  x + ky &= 1, \\
  kx + y &= 1,
\end{align*}
has
\begin{enumerate}[(a)]
\item no solution,
\item a unique solution, and
\item infinitely many solutions.
\item When there is exactly one solution, what are the values of $x$ and $y$. \\
\end{enumerate}

\noindent{\bfseries Question 3.} Consider the following two systems of
equations.
\begin{align*}
  x+y+z &= 16, \\
  x+2y+2z &= 11, \\
  2x+3y-4z &= 3,
\end{align*}
and
\begin{align*}
  x+y+z &= 7, \\
  x+2y+2z &= 10, \\
  2x+3y-4z &= 3.
\end{align*}
Solve both systems simultaneously by applying row reduction to an appropriate $3
\times 5$ matrix. \\

\noindent{\bfseries Question 4.} Consider the following homogeneous system of
linear equations where $a,b \in \R$ are constants.
\begin{align*}
  x+2y &= 0, \\
  ax+8y+3z &= 0, \\
  by+5z &= 0.
\end{align*}
\begin{enumerate}[(a)]
\item Find a value for $a$ which makes it necessary to interchange rows during
  row reduction.
\item Suppose that $a$ does not have the value you found in part (a). Find a
  value for $b$ so that the system has a nontrivial solution.
\item Suppose that $a$ does not have the value you found in part (a) and that
  $b=100$. Suppose further that $a$ is chosen so that the solution to the system
  is not unique. The general solution to the system is
  $(\alpha^{-1}z,-\beta^{-1}z,z)$ where $\alpha$ and $\beta$ are what? \\
\end{enumerate}

\subsection*{Spans of Collections of Vectors}

\noindent{\bfseries Question 5.} Consider the following three vectors in $\R^3$
\[
  \vec u_1 = \begin{pmatrix}1\\1\\0\end{pmatrix},\quad
  \vec u_2 = \begin{pmatrix}0\\1\\1\end{pmatrix},\quad
  \vec u_3 = \begin{pmatrix}1\\0\\1\end{pmatrix}.
\]
Show that $\R^3=\text{span}\{\vec u_1,\,\vec u_2,\,\vec u_3\}$. \\

\noindent{\bfseries Question 6.} Consider the following four vectors in $\R^4$
given by
\[
  \vec v_1 = \begin{pmatrix}+1 \\ -1 \\ -1 \\ -1\end{pmatrix},  \quad
  \vec v_2 = \begin{pmatrix}-1 \\ +1 \\ -1 \\ -1\end{pmatrix}, \quad
  \vec v_3 = \begin{pmatrix}-1 \\ -1 \\ +1 \\ -1\end{pmatrix}, \quad
  \vec v_4 = \begin{pmatrix}-1 \\ -1 \\ -1 \\ +1\end{pmatrix}.
\]
\begin{enumerate}[(a)]
\item\label{Q1: span} Show whether $\vec v_1 \in \text{span}\{\vec v_2,\,\vec
  v_3,\,\vec v_4\}$ 
  or not by solving the corresponding system of linear equations.
\item\label{Q1: lin indep} Let $a_1,\,a_2,\,a_3,\,a_4 \in \R$. Under what
  conditions on $a_1,\,a_2,\,a_3,\,a_4$ is $a_1\vec v_1+a_2\vec v_2+a_3\vec
  v_3+a_4\vec v_4=\vec{0}$ true?
\item How can we use part \ref{Q1: lin indep} to provide a second proof of part
  \ref{Q1: span}? Can you generalize to answer the following question: Is $\vec
  v_i \in \text{span}\{\vec v_j,\,\vec v_k,\,\vec v_l\}$ for $i,\,j,\,k,\,l$
  distinct? \\
\end{enumerate}

\noindent{\bfseries Question 7.} Let $V_1$ and $V_2$ be two subsets of $\R^n$, and
define $V_1+V_2=\{\vec v_1 + \vec v_2 : \vec v_1 \in V_1 \text{ and } \vec v_2
\in V_2 \}$. Show
\begin{enumerate*}[(a)]
\item $\text{span}(V_1 \cup V_2) = \text{span}(V_1) + \text{span}(V_2)$, and
\item $\text{span}(V_1 \cap V_2) \subseteqq \text{span}(V_1) \cap
  \text{span}(V_2)$.
\end{enumerate*}
Further, give an example of subsets $V_1$ and $V_2$ of $\R^n$, for some $n$, for
which $\text{span}(V_1 \cap V_2) \subsetneqq \text{span}(V_1) \cap
\text{span}(V_2)$. \\

\subsection*{Linear Independence}

\noindent{\bfseries Question 8.} Show that in $\R^3$, the vectors $\vec
x=(1,1,0)$, $\vec y=(0,1,2)$, and $\vec z=(3,1,-4)$ are linearly dependent by
finding scalars $\alpha$ and $\beta$ such that $\alpha\vec x+\beta\vec y+\vec
z=\vec 0$. \\

\noindent{\bfseries Question 9.} Let $\vec w=(1,1,0,0)$, $\vec x=(1,0,1,0)$, $\vec
y=(0,0,1,1)$, and $\vec z=(0,1,0,1)$, and let $S=\{\vec w,\,\vec x,\,\vec
y,\,\vec z\}$.
\begin{enumerate}[(a)]
\item Show that $S$ is not a spanning set for $\R^4$ by finding a vector $\vec
  u$ in $\R^4$ such that $\vec u \notin\text{span}(S)$. One such vector is $\vec
  u=(1,2,3,a)$ where $a$ is any real number execpt what?
\item Show that $S$ is a linearly dependent set of vectors by finding scalars
  $\alpha$, $\gamma$, and $\delta$ such that $\alpha\vec w+\vec x+\gamma\vec
  y+\delta\vec z=\vec 0$.
\item Show that $S$ is a linear dependent set by writing $\vec z$ as a linear
  combination of the remaining vectors in $S$.
\end{enumerate}

\noindent{\bfseries Question 10.} Let $S_1$ and $S_2$ be finite subsets of $\R^n$,
for some $n$, such that $S_1 \subseteqq S_2$. Prove that if $S_1$ is a linearly
dependent set, then so is $S_2$. Show that this is equivalent to if $S_2$ is a
linearly independent set, then so is $S_1$. \\

\noindent{\bfseries Question 11.} Do the following.
\begin{enumerate}[(a)]
\item Let $\vec u$ and $\vec v$ be distinct vectors in $\R^n$. Prove that
  $\{\vec u, \vec v\}$ is linearly independent if and only if $\{\vec u + \vec v,
  \vec u - \vec v\}$ is linearly independent.
\item Let $\vec u, \vec v, \vec w$ be distinct vectors in $\R^n$. Prove that
  $\{\vec u, \vec v, \vec w\}$ is linearly independent if and only if $\{\vec
  u+\vec v,\vec u+\vec w,\vec v+\vec w\}$ is linear independent. \\
\end{enumerate}

\subsection*{Linear Transformations}

\noindent{\bfseries Question 12.} Show the following for $\R^n$.
\begin{enumerate}[(a)]
\item Show that scalar multiplication is a linear transformation.
\item When is this linear map invertible?
\item Is its inverse a linear transformation?
\item Fix an element $a \in \R^n$. What is the matrix corresponding to the
linear transformation $\vec v \mapsto a\vec v$? \\
\end{enumerate}

\noindent{\bfseries Question 13.} Fix $a \in \R$ and $\vec u \in \R^n$ with $\vec
u \neq \vec 0$. Is the map given by $\vec v \mapsto a\vec v + \vec u$, linear?
Why or why not? \\

\noindent{\bfseries Question 14.} Consider a linear transformation $T: \R^n
\rightarrow \R^n$, and define $\text{Ker}(T)=\{\vec v \in \R^n : T(\vec v)=\vec
0\}$. This is the kernel of the linear transformation $T$. For $\vec v \in
\R^n$, define $\vec v + \text{Ker}(T)=\{\vec v + \vec u : \vec u \in
\text{Ker}(T)\}$. Show the following.
\begin{enumerate}[(a)]
\item $\text{Ker}(T)$ is closed under scalar multiplcation and vector addition.
\item For $\vec v \in \R^n$, show that $\vec v + \text{Ker}(T)$ consists of all
  and only those elements of $\R^n$ that map to $T(\vec v)$ under $T$.
\item For $\vec v_1,\vec v_2 \in \R^n$, show that either $\vec
  v_1+\text{Ker}(T)=\vec v_2 +\text{Ker}(T)$ or $\vec v_1+\text{Ker}(T) \cap
  \vec v_2 +\text{Ker}(T)=\emptyset$.
\end{enumerate}

\subsection*{Matrix Operations}

\noindent{\bfseries Question 15.} Give an example of a nonzero matrix $A$ such
that $A^2=O$. \\

\noindent{\bfseries Question 16.} The trace of a square matrix $A$ of dimensions
$N \times N$ is defined as $\text{tr}(A)=\sum_{k=1}^NA_{k,k}$, i.e., the sum of
the diagonal entries of the matrix. For any other $N \times N$ matrix $B$, show
that $\text{tr}(AB)=\text{tr}(BA)$. \\

\noindent{\bfseries Question 17.} An $N \times N$ matrix $A$ is circulant if it is
of the form
\[
  A=\begin{pmatrix}
      a_1 & a_2 & a_3 & \cdots & a_N \\
      a_N & a_1 & a_2 & \cdots & a_{N-1} \\
      a_{N-1} & a_N & a_1 & \cdots & a_{N-2} \\
      \vdots & \vdots & \vdots & \ddots & \vdots \\
      a_2 & a_3 & a_4 & \cdots & a_1
    \end{pmatrix}.
\]
Show that if $B$ is any other $N \times N$ circulant matrix, then $AB=BA$. \\

\noindent{\bfseries Question 18.} A diagonal matrix is one for which every entry
not on the main diagonal is zero. Let $A$ and $B$ be $N \times N$ matrices such
that there exists and invertible $N \times N$ matrix $P$ for which
$D_A=P^{-1}AP$ and $D_B=P^{-1}BP$ are diagonal matrices. Show that $A$ and $B$
commute.

\end{document}