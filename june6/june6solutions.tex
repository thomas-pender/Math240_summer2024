\documentclass[a4paper,11pt]{article}

\usepackage[utf8]{inputenc}
\usepackage[english]{babel}
\usepackage{amssymb, amsmath, amsthm, mathrsfs}
\usepackage[left=1.0in,right=1.0in,top=1.0in,bottom=1.0in]{geometry}
\usepackage[inline,shortlabels]{enumitem}
\usepackage{times}
\usepackage{xcolor}

\newcommand{\R}{\mathbf{R}}
\newcommand{\BB}[1]{\textcolor{blue}{#1}}
\newcommand{\abs}[1]{\vert #1 \vert}
\newcommand{\bb}{\backslash}

\begin{document}

\begin{center}
  {\Large\bfseries Math 240 Tutorial \\ Solutions}
\end{center}
\begin{center}
  {\bfseries June 6}
\end{center}

\subsection*{Systems of Linear Equations and Row Reduction}

\noindent{\bfseries Question 1.} Place the following augmented matrices into
an echelon form. Does the corresponding system of linear equations admit any
solutions?
\begin{enumerate}[(a)]
\item
  \[
    \left(
      \begin{array}{cccc|c}
        4 & 8 & 12 & 4 & 7 \\
        2 & 5 & 6 & 6 & 11 \\
        0 & 5 & 1 & 26 & 13 \\
        0 & 5 & 0 & 21 & 17
      \end{array}
    \right).
  \]

  \BB{
    An echelon form:
    \[
      \left(
        \begin{array}{cccc|c}
          2 & 5 & 6 & 6 & 11 \\
          0 & 2 & 0 & 8 & 15 \\
          0 & 0 & 1 & 5 & -4 \\
          0 & 0 & 0 & 2 & -41
        \end{array}
      \right).
    \]
    Every column except the last is a pivot column, so the system has a unique
    solution.  \\
  }
  
\item
  \[
    \left(
      \begin{array}{cccc|c}
        4 & 8 & 12 & 4 & 0 \\
        2 & 5 & 6 & 6 & 0 \\
        0 & 5 & 1 & 25 & 0 \\
        0 & 5 & 0 & 20 & 0       
      \end{array}
    \right).
  \]

  \BB{
    An echelon form:
    \[
      \left(
        \begin{array}{cccc|c}
          2 & 5 & 6 & 6 & 0 \\
          0 & 1 & 0 & 4 & 0 \\
          0 & 0 & 1 & 5 & 0 \\
          0 & 0 & 0 & 0 & 0
        \end{array}
      \right).
    \]
    The corresponding system has infinitely many solutions. \\
  }

\item
  \[
    \left(
      \begin{array}{cccc|c}
        4 & 8 & 12 & 4 & 7 \\
        2 & 5 & 6 & 6 & 11 \\
        0 & 5 & 1 & 25 & 13 \\
        0 & 5 & 0 & 20 & 17       
      \end{array}
    \right).
  \]

  \BB{
    An echelon form:
    \[
      \left(
        \begin{array}{cccc|c}
          2 & 5 & 6 & 6 & 11 \\
          0 & 2 & 0 & 8 & 15 \\
          0 & 0 & 1 & 5 & -4 \\
          0 & 0 & 0 & 0 & 41
        \end{array}
      \right).
    \]
    Since the last column is a pivot column, the corresponding system of linear
    equations is inconsistent. \\
  }
\end{enumerate}

\noindent{\bfseries Question 2.} Find the values of $k$ for which the system of
equations
\begin{align*}
  x + ky &= 1, \\
  kx + y &= 1,
\end{align*}
has
\begin{enumerate}[(a)]
\item no solution, \\

  \BB{
    The augmented matrix for the system of linear equations is given by
    \[
      \left(
        \begin{array}{rr|r}
          1 & k & 1 \\
          k & 1 & 1
        \end{array}
      \right).
    \]
    If $k=0$, then there is a unique solution given by $(x,y)=(1,1)$; so, we
    assume that $k \neq 0$. The Gaussian form of the matrix is then
    \[
      \left(
        \begin{array}{rr|r}
          1 & k & 1 \\
          0 & \frac{1}{k}-k & \frac{1}{k}-1
        \end{array}
      \right)
    \]
    If $\frac{1}{k}-k \neq 0$, that is, if $\abs{k} \neq 1$, then there is a
    unique solution given by $(x,y) = ((1+k)^{-1},(1+k)^{-1})$.
  }

  \BB{
    It remains to examine the case that $\abs{k}=1$. If $k=1$, then we have the
    equation $x+y=1$. This has infinitely many solutions. If $k=-1$, then we
    have the have the system
    \begin{align*}
      x-y &= 1, \\
      -x+y &= 1.
    \end{align*}
    Adding these two equations gives $0=2$, a contradiction. Therefore, there
    is no solution only in the case that $k=-1$. \\
  }
  
\item a unique solution, and \\

  \BB{
    From our work in part (a), there is a unique solution whenever $\abs{k} \neq
    1$. \\
  }

\item infinitely many solutions. \\

  \BB{
    From our work in part (a), there are infinitely many solutions in the case
    that $k=1$. \\
  }
  
\item When there is exactly one solution, what are the values of $x$ and $y$. \\

  \BB{
    By part (a), this happens whenever $\abs{k} \neq 1$. If $k=0$, then
    $(x,y)=(1,1)$. If $k \neq 0$ and $\abs{k} \neq 1$, then 
    \[
      (x,y) = \left( \frac{1}{1+k}, \frac{1}{1+k} \right).
    \] \\
  }
\end{enumerate}

\noindent{\bfseries Question 3.} Consider the following two systems of
equations.
\begin{align*}
  x+y+z &= 16, \\
  x+2y+2z &= 11, \\
  2x+3y-4z &= 3,
\end{align*}
and
\begin{align*}
  x+y+z &= 7, \\
  x+2y+2z &= 10, \\
  2x+3y-4z &= 3.
\end{align*}
Solve both systems simultaneously by applying row reduction to an appropriate $3
\times 5$ matrix. \\

\BB{We consider the following matrix augmented by two columns
  \[
    \left(
      \begin{array}{rrr|rr}
        1 & 1 & 1 & 16 & 7 \\
        1 & 2 & 2 & 11 & 10 \\
        2 & 3 & -4 & 3 & 3
      \end{array}
    \right).
  \]
  Its reduced row echelon form is given by
  \[
    \left(
      \begin{array}{rrr|rr}
        1 & 0 & 0 & 21 & 4 \\
        0 & 1 & 0 & -\frac{59}{7} & 1 \\
        0 & 0 & 1 & \frac{24}{7} & 2
      \end{array}
    \right).
  \]
  Therefore, both systems have a unique solution. The first is given by $x=21$,
  $y=-59/7$, and $z=24/7$; and the second is given by $x=4$, $y=1$, and $z=2$. \\
}

\noindent{\bfseries Question 4.} Consider the following homogeneous system of
linear equations where $a,b \in \R$ are constants.
\begin{align*}
  x+2y &= 0, \\
  ax+8y+3z &= 0, \\
  by+5z &= 0.
\end{align*}
\begin{enumerate}[(a)]
\item Find a value for $a$ which makes it necessary to interchange rows during
  row reduction. \\

  \BB{The augmented matrix for the system is
    \[
      \left(
        \begin{array}{rrr|r}
          1 & 2 & 0 & 0 \\
          a & 8 & 3 & 0 \\
          0 & b & 5 & 0
        \end{array}
      \right).
    \]
    If $a=4$, then $-4$ times the first row added to the second gives
    \[
      \left(
        \begin{array}{rrr|r}
          1 & 2 & 0 & 0 \\
          0 & 0 & 3 & 0 \\
          0 & b & 5 & 0
        \end{array}
      \right)
    \]
    Therefore, if $b \neq 0$, then we would be required to interchange the
    second and third rows. It isn't difficult to see that this is the only case
    in which we would need to interchange rows. \\
  }

\item Suppose that $a$ does not have the value you found in part (a). Find a
  value for $b$ so that the system has a nontrivial solution. \\

  \BB{Since $a \neq 4$, we have
    \[
      \left(
        \begin{array}{rrr|r}
          1 & 2 & 0 & 0 \\
          a & 8 & 3 & 0 \\
          0 & b & 5 & 0
        \end{array}
      \right)\sim
      \left(
        \begin{array}{rrr|r}
          1 & 2 & 0 & 0 \\
          0 & 1 & \frac{3}{8-2a} & 0 \\
          0 & 0 & 5-\frac{3b}{8-2a} & 0
        \end{array}
      \right).
    \]
    This system has a nontrivial solution only in the case $5-3b/(8-2a)=0$, that
    is, only in the case $b=(40-10a)/3$. \\
  }
  
\item Suppose that $a$ does not have the value you found in part (a) and that
  $b=100$. Suppose further that $a$ is chosen so that the solution to the system
  is not unique. The general solution to the system is
  $(\alpha^{-1}z,-\beta^{-1}z,z)$ where $\alpha$ and $\beta$ are what? \\

  \BB{Since we assume that $a \neq 4$, we can row reduce the coefficient matrix
    to
    \[
      \left(
        \begin{array}{rrr}
          1 & 2 & 0 \\
          0 & 1 & \frac{3}{8-2a} \\
          0 & 0 & 5-\frac{300}{8-2a}
        \end{array}
      \right).
    \]
    In order for the system to have nontrivial solutions, we require
    $5-\frac{300}{8-2a}=0$, that is, $a=-26$. Therefore, the general solution is
    given by
    \[
      x=\frac{1}{10}z, \qquad y=-\frac{1}{20}z
    \]
    where $z$ is free. So, $\alpha=10$ and $\beta=-20$. \\
  }
\end{enumerate}

\subsection*{Spans of Collections of Vectors}

\noindent{\bfseries Question 5.} Consider the following three vectors in $\R^3$
\[
  \vec u_1 = \begin{pmatrix}1\\1\\0\end{pmatrix},\quad
  \vec u_2 = \begin{pmatrix}0\\1\\1\end{pmatrix},\quad
  \vec u_3 = \begin{pmatrix}1\\0\\1\end{pmatrix}.
\]
Show that $\R^3=\text{span}\{\vec u_1,\,\vec u_2,\,\vec u_3\}$. \\

\BB{We will prove this result in two ways. First, observing that
  \[
    \left(
      \begin{array}{rrr}
        1 & 0 & 1 \\
        1 & 1 & 0 \\
        0 & 1 & 1
      \end{array}
    \right)
    \sim
    \left(
      \begin{array}{rrr}
        1 & 0 & 0 \\
        0 & 1 & 0 \\
        0 & 0 & 1
      \end{array}
    \right)
  \]
  the result follows. If, however, we required more information in how a vector
  of $R^3$ is decomposable as a linear combination of $\vec u_1,\,\vec u_2,\,\vec
  u_3$, we could note the following
  \[
    \left(
      \begin{array}{rrrr}
        1 & 0 & 1 & a_1 \\
        1 & 1 & 0 & a_2 \\
        0 & 1 & 1 & a_3
      \end{array}
    \right) 
    \sim
    \left(
      \begin{array}{rrrr}
        1 & 0 & 0 & \frac{1}{2}a_1+\frac{1}{2}a_2-\frac{1}{2}a_3 \\
        0 & 1 & 0 & -\frac{1}{2}a_1+\frac{1}{2}a_2+\frac{1}{2}a_3 \\
        0 & 0 & 1 & \frac{1}{2}a_1-\frac{1}{2}a_2+\frac{1}{2}a_3
      \end{array}
    \right).
  \]
  This means that if
  $
  \left(
    \begin{smallmatrix}
      a_1\\a_2\\a_3
    \end{smallmatrix}
  \right)
  \in \R^3,
  $
  then
  \[
    \left( \frac{1}{2}a_1+\frac{1}{2}a_2-\frac{1}{2}a_3 \right)\vec u_1-
    \left( \frac{1}{2}a_1-\frac{1}{2}a_2-\frac{1}{2}a_3 \right)\vec u_2+
    \left( \frac{1}{2}a_1-\frac{1}{2}a_2+\frac{1}{2}a_3 \right)\vec u_3=
    \begin{pmatrix}a_1\\a_2\\a_3\end{pmatrix}.
  \] \\
}

\noindent{\bfseries Question 6.} Consider the following four vectors in $\R^4$
given by
\[
  \vec v_1 = \begin{pmatrix}+1 \\ -1 \\ -1 \\ -1\end{pmatrix},  \quad
  \vec v_2 = \begin{pmatrix}-1 \\ +1 \\ -1 \\ -1\end{pmatrix}, \quad
  \vec v_3 = \begin{pmatrix}-1 \\ -1 \\ +1 \\ -1\end{pmatrix}, \quad
  \vec v_4 = \begin{pmatrix}-1 \\ -1 \\ -1 \\ +1\end{pmatrix}.
\]
\begin{enumerate}[(a)]
\item\label{Q1: span} Show whether $\vec v_1 \in \text{span}\{\vec v_2,\,\vec
  v_3,\,\vec v_4\}$ 
  or not by solving the corresponding system of linear equations. \\

  \BB{Observe
    \[
      \left(
        \begin{array}{rrr|r}
          -1 & -1 & -1 & 1 \\
          1 & -1 & -1 & -1 \\
          -1 & 1 & -1 & -1 \\
          -1 & -1 & 1 & -1
        \end{array}
      \right) 
      \sim
      \left(
        \begin{array}{rrr|r}
          1 & 0 & 0 & 0 \\
          0 & 1 & 0 & 0 \\
          0 & 0 & 1 & 0 \\
          0 & 0 & 0 & 1
        \end{array}
      \right)
    \] so that $\vec v_1 \notin \text{span}\{\vec v_2,\,\vec v_3,\,\vec v_4\}$. \\
  }
  
\item\label{Q1: lin indep} Let $a_1,\,a_2,\,a_3,\,a_4 \in \R$. Under what
  conditions on $a_1,\,a_2,\,a_3,\,a_4$ is $a_1\vec v_1+a_2\vec v_2+a_3\vec
  v_3+a_4\vec v_4=\vec{0}$ true? \\

  \BB{We have
    \[
      \left(
        \begin{array}{rrrr}
          1 & -1 & -1 & -1 \\
          -1 & 1 & -1 & -1 \\
          -1 & -1 & 1 & -1 \\
          -1 & -1 & -1 & 1
        \end{array}
      \right) 
      \sim
      \left(
        \begin{array}{rrrr}
          1 & 0 & 0 & 0 \\
          0 & 1 & 0 & 0 \\
          0 & 0 & 1 & 0 \\
          0 & 0 & 0 & 1
        \end{array}
      \right)
    \]
    so that only the solution $a_1=a_2=a_3=a_4=0$ exists. \\
  }
  
\item How can we use part \ref{Q1: lin indep} to provide a second proof of part
  \ref{Q1: span}? Can you generalize to answer the following question: Is $\vec
  v_i \in \text{span}\{\vec v_j,\,\vec v_k,\,\vec v_l\}$ for $i,\,j,\,k,\,l$
  distinct? \\

  \BB{We show only the second question (the first being a special case of the
    second). We will use proof by contradiction. Assume to the contrary that
    there exists $a_j,\,a_k,\,a_l \in \R$ not all zero such that $\vec v_i
    = a_j\vec v_j + a_k\vec v_k + a_l\vec v_l$. But then $\vec v_i - a_j\vec v_j
    - a_k\vec v_k - a_l\vec v_l = \vec 0$. From our answer to part \ref{Q1: lin
      indep}, it follows that $a_j=a_k=a_l=0$ and $1=0$. But $1=0$ is a
    contradiction. Therefore, our original assumption that $\vec v_i \in
    \text{span}\{\vec v_j,\,\vec v_k,\,\vec v_l\}$ is incorrect. That is, we
    must have $\vec v_i \notin \text{span}\{\vec v_j,\,\vec v_k,\,\vec v_l\}$ \\}
\end{enumerate}

\noindent{\bfseries Question 7.} Let $V_1$ and $V_2$ be two subsets of $\R^n$, and
define $V_1+V_2=\{\vec v_1 + \vec v_2 : \vec v_1 \in V_1 \text{ and } \vec v_2
\in V_2 \}$. Show
\begin{enumerate*}[(a)]
\item $\text{span}(V_1 \cup V_2) = \text{span}(V_1) + \text{span}(V_2)$, and
\item $\text{span}(V_1 \cap V_2) \subseteqq \text{span}(V_1) \cap
  \text{span}(V_2)$.
\end{enumerate*}
Further, give an example of subsets $V_1$ and $V_2$ of $\R^n$, for some $n$, for
which $\text{span}(V_1 \cap V_2) \subsetneqq \text{span}(V_1) \cap
\text{span}(V_2)$. \\

\BB{
  Throughout, let $V_1 \cap V_2 = \{w_1,\,w_2,\dots,\,w_p\}$,
  $V_1=\{\vec v_1,\,\vec v_2,\dots,\,\vec v_{n-p},\,\vec w_1,\,\vec
  w_2,\dots,\,\vec w_p\}$, $V_2=\{\vec u_1,\,\vec u_2,\dots,\,\vec
  u_{m-p},\,\vec w_1,\,\vec w_2,\dots,\,\vec w_p\}$.
  \begin{enumerate}[(a)]
  \item Let $\vec x \in \text{span}(V_1 \cup V_2)$. Then there exists scalars
    $a_1,\,a_2,\dots,\,a_{n-p} \in \R$ and $b_1,\,b_2,\dots,\,b_{m-p} \in \R$
    and $c_1,\,c_2,\dots,\,c_p \in \R$ such that 
    \[
      \vec x = \sum_{i=1}^{n-p}a_i\vec v_i + \sum_{j=1}^{m-p}b_j\vec u_j +
      \sum_{k=1}^pc_k\vec w_k.
    \]
    But
    \[
      \sum_{i=1}^{n-p}a_i\vec v_i + \sum_{k=1}^pc_k\vec w_k \in \text{span}(V_1)
    \]
    and
    \[
      \sum_{j=1}^{m-p}b_j\vec u_j \in \text{span}(V_2).
    \]
    Therefore, $\vec x \in \text{span}(V_1)+\text{span}(V_2)$, and
    $\text{span}(V_1 \cup V_2) \subseteqq \text{span}(V_1)+\text{span}(V_2)$.
    Conversely, let $\vec y \in \text{span}(V_1)+\text{span}(V_2)$. Then there
    exists scalars $\alpha_1,\,\alpha_2,\dots,\,\alpha_n \in \R$ and
    $\beta_1,\,\beta_2,\dots,\,\beta_m \in \R$ such that
    \[
      y = \sum_{i=1}^{n-p}\alpha_i\vec v_i +
      \sum_{j=1}^p(\alpha_j+\beta_j)\vec w_j +
      \sum_{k=1}^{m-p}\beta_k\vec u_k.
    \]
    Since $V_1 \cup V_2=\{\vec v_1,\,\dots,\,\vec v_n,\,\vec u_1,\dots,\,\vec
    u_m,\,\vec w_1,\dots,\,\vec w_p\}$, we have that $\vec y \in \text{span}(V_1
    \cup V_2)$, and $\text{span}(V_1 \cup V_2) \supseteqq
    \text{span}(V_1)+\text{span}(V_2)$.
  \item Let $\vec x \in \text{span}(V_1 \cap V_2)$. Then there exists scalars
    $a_1,\,a_2,\dots,\,a_p$ such that
    \[
      \vec x = a_1\vec w_1 + a_2\vec w_2 + \cdots + a_p\vec w_p.
    \]
    Since $V_1 \cap V_2 \subseteqq V_1$ and $V_1 \cap V_2 \subseteqq V_2$, we
    see at once that $\vec x$ is in both $\text{span}(V_1)$ and
    $\text{span}(V_2)$, that is, $\vec x \in
    \text{span}(V_1)\cap\text{span}(V_2)$, as desired. Any number of counter
    examples can be found to show that $\text{span}(V_1 \cap V_2) =
    \text{span}(V_1) \cap \text{span}(V_2)$ is not true in general. It isn't
    difficult to see that $V_1=\left\{
      \left( \begin{smallmatrix}1\\0\end{smallmatrix}
      \right),\,\left( \begin{smallmatrix}0\\1\end{smallmatrix} \right)
    \right\}$  and $V_2=\left\{
      \left( \begin{smallmatrix}1\\1\end{smallmatrix} \right) \right\}$ give
    a counter example. Since then $\text{span}(V_1 \cap V_2)=\{\vec 0\}$ and
    $\text{span}(V_1)\cap\text{span}(V_2)=\text{span}(V_2) \neq \{\vec 0\}$. \\
  \end{enumerate}}

\subsection*{Linear Independence}

\noindent{\bfseries Question 8.} Show that in $\R^3$, the vectors $\vec
x=(1,1,0)$, $\vec y=(0,1,2)$, and $\vec z=(3,1,-4)$ are linearly dependent by
finding scalars $\alpha$ and $\beta$ such that $\alpha\vec x+\beta\vec y+\vec
z=\vec 0$. \\

\BB{We solve the system $\alpha\vec x+\beta\vec y=-\vec z$ to find $\alpha=-3$
  and $\beta=2$. \\}

\noindent{\bfseries Question 9.} Let $\vec w=(1,1,0,0)$, $\vec x=(1,0,1,0)$, $\vec
y=(0,0,1,1)$, and $\vec z=(0,1,0,1)$, and let $S=\{\vec w,\,\vec x,\,\vec
y,\,\vec z\}$.
\begin{enumerate}[(a)]
\item Show that $S$ is not a spanning set for $\R^4$ by finding a vector $\vec
  u$ in $\R^4$ such that $\vec u \notin\text{span}(S)$. One such vector is $\vec
  u=(1,2,3,a)$ where $a$ is any real number execpt what? \\

  \BB{We form the matrix $A$ whose columns are $\vec w$, $\vec x$, $\vec y$, and
    $\vec z$. The reduced row echelon form for $A$ is then
    \[
      \left(
        \begin{array}{rrrr}
          1 & 0 & 0 & 1 \\
          0 & 1 & 0 & -1 \\
          0 & 0 & 1 & 1 \\
          0 & 0 & 0 & 0
        \end{array}
      \right)
    \]
    whereupon we see that $\text{span}\{\vec w,\,\vec x,\,\vec y,\,\vec
    z\}=\text{span}\{\vec w,\,\vec x,\,\vec y\}$. We can then check that the
    system
    \[
      \alpha\vec w+\beta\vec x+\gamma\vec y=
      \begin{pmatrix}
        1\\2\\3\\a
      \end{pmatrix}
    \]
    is inconsistent precisely in the case that $a=4$. Therefore, the vector
    $(1,2,3,4)^t$ is not in the span of $\vec w$, $\vec x$, $\vec y$, and
    $\vec z$ so that they do not span $\R^4$. \\}
  
\item Show that $S$ is a linearly dependent set of vectors by finding scalars
  $\alpha$, $\gamma$, and $\delta$ such that $\alpha\vec w+\vec x+\gamma\vec
  y+\delta\vec z=\vec 0$. \\

  \BB{Solving the system $\alpha\vec w+\gamma\vec y+\delta\vec z=-\vec x$, we
    find that there is a unique solution given by $\alpha=\gamma=-1$ and
    $\delta=1$. \\}
  
\item Show that $S$ is a linear dependent set by writing $\vec z$ as a linear
  combination of the remaining vectors in $S$. \\

  \BB{From the reduced row echelon form for the matrix $A$ in part (a), we see
    that $\vec z=\vec w-\vec x+\vec y$. \\}
\end{enumerate}

\noindent{\bfseries Question 10.} Let $S_1$ and $S_2$ be finite subsets of $\R^n$,
for some $n$, such that $S_1 \subseteqq S_2$. Prove that if $S_1$ is a linearly
dependent set, then so is $S_2$. Show that this is equivalent to if $S_2$ is a
linearly independent set, then so is $S_1$. \\

\BB{Let $S_1$ and $S_2$ be finite subsets of $\R^n$ such that $S_1 \subseteqq
  S_2$. Suppose that $S_1$ is linearly dependent, but $S_2$ is linearly
  independent. For definiteness, we write
  \[
    S_1 = \{\vec v_1,\,\vec v_2,\dots,\,\vec v_m\}
  \]
  and
  \[
    S_2 = \{\vec v_1,\,\vec v_2,\dots,\,\vec v_m,\,\vec v_{m+1},\dots,\,\vec v_k\}.
  \] By assumption, there is a collection $a_1,\,a_2,\dots,\,a_m$ of scalars such
  that $a_1\vec v_1 + a_2\vec v_2 + \cdots + a_m\vec v_m=\vec 0$ but $a_i \neq 0$
  for some $i \in \{1,\,2,\dots,\,m\}$. Set $a_{m+1}=a_{m+2}=\cdots=a_k=0$ so that
  \[
    a_1\vec v_1 + a_2\vec v_2 + \cdots + a_m\vec v_m + a_{m+1}\vec v_{m+1} + \cdots
    + a_k\vec v_k = \vec 0.
  \]
  By our assumption that $S_2$ is linear independent, we have
  \[
    a_1=a_2=\cdots=a_m=a_{m+1}=\cdots=a_k=0
  \]
  by the definition of linearly independent. But this implies that each $a_i=0$ for
  $i \in \{1,\,2,\dots,\,m\}$, contrary to what we have observed. This
  contradiction proves the result. \\}

\noindent{\bfseries Question 11.} Do the following.
\begin{enumerate}[(a)]
\item Let $\vec u$ and $\vec v$ be distinct vectors in $\R^n$. Prove that
  $\{\vec u, \vec v\}$ is linearly independent if and only if $\{\vec u + \vec v,
  \vec u - \vec v\}$ is linearly independent. \\

  \BB{ Suppose first that $\{\vec u,\,\vec v\}$ is linearly independent, but
$\vec u + \vec v = a(\vec u - \vec v)$ for some $a \in \R$. If $a=1$, then $\vec
v=\vec 0$, contradicting the fact that $\{\vec u,\,\vec v\}$ is independent. So
$a \neq 1$ and $\vec u = -\frac{1+a}{1-a}\vec v$, again, contradicting the fact
that $\{\vec u,\,\vec v\}$ is independent.}

\BB{Conversely, suppose that $\vec u = b\vec v$ for some $b \in \R$. Then $\vec
u + \vec v = (1+b)\vec u$ and $\vec u - \vec v = (1-b)\vec u$. But this means
that both $\vec u + \vec v$ and $\vec u - \vec v$ are in $\text{span}\{\vec
u\}$, whereupon $\{\vec u + \vec v, \vec u - \vec v\}$ is linearly dependent.
\\}
  
\item Let $\vec u, \vec v, \vec w$ be distinct vectors in $\R^n$. Prove that
  $\{\vec u, \vec v, \vec w\}$ is linearly independent if and only if $\{\vec
  u+\vec v,\vec u+\vec w,\vec v+\vec w\}$ is linear independent. \\

  \BB{First, we assume that $\{\vec u,\,\vec v,\,\vec w\}$ is linearly
    independent, and we assume there are scalars $a,\,b,\,c$ such that
    \[
      \vec 0 = a(\vec u+\vec v)+b(\vec u+\vec w)+c(\vec v+\vec w)
      = (a+b)\vec u+(a+c)\vec v+(b+c)\vec w.
    \]
    Since we are assuming that $\{\vec u,\,\vec v,\,\vec w\}$ is linearly
    independent, it must be that $a+b=a+c=b+c=0$. This linear system has only
    the trivial solution $a=b=c=0$, whereupon $\{\vec u+\vec v,\vec u+\vec
    w,\vec v+\vec w\}$ is linearly independent as well.
  }

  \BB{Next, we assume that $\{\vec u+\vec v,\vec u+\vec w,\vec v+\vec w\}$ is
    independent. Suppose there are scalars $a,\,b,\,c$ such that $a\vec u+b\vec
    v+c\vec w=\vec 0$. Then
    \[
      (a-b+c)(\vec u+\vec v)+(a+b-c)(\vec u+\vec w)+(-a+b+c)(\vec v+\vec w)=\vec 0.
    \]
    By our assumed linear independence of $\{\vec u+\vec v,\vec u+\vec w,\vec
    v+\vec w\}$, we have $a-b+c=a+b-c=-a+b+c=0$. Solving, we find this possesses
    only the trivial solution $a=b=c=0$. It follows $\{\vec u,\,\vec v,\,\vec
    w\}$ is independent as well. \\}
\end{enumerate}

\subsection*{Linear Transformations}

\noindent{\bfseries Question 12.} Show the following for $\R^n$.
\begin{enumerate}[(a)]
\item Show that scalar multiplication is a linear transformation. \\

  \BB{Fix $a \in \R$, and let $T:\R^n \rightarrow \R^n$ be the map $T(\vec
    v)=a\vec v$. Then $T(b\vec v)=ab\vec v=ba\vec v=bT(\vec v)$ and $T(\vec
    v+\vec u)=a(\vec v+\vec u)=a\vec v+a\vec u=T(\vec v)+T(\vec u)$. We have
    shown that $T$ is linear. \\}

\item When is this linear map invertible? \\

  \BB{This map is invertible precisely in the case the scalar by which we are
    multiplying is nonzero. \\}
  
\item Is its inverse a linear transformation? \\

  \BB{Let $T$ be as in part (a), and assume that $a \neq 0$. Then $T$ is
    invertible and $T^{-1}$ is given by multiplcation by $a^{-1}$. Since this
    is multiplcation by a scalar, it is linear. \\}
  
\item Fix an element $a \in \R^n$. What is the matrix corresponding to the
linear transformation $\vec v \mapsto a\vec v$ with respect to the standard
spanning vectors? \\

\BB{Recall the standard spanning vectors are $\{\vec e_1,\,\vec e_2,\dots,\,\vec
  e_n\}$ where $\vec e_i$ is the vector with a 1 in position $i$ and zeros
  everywhere else. Then the matrix corresponding to multiplication by $a$ is
  givn by
  \[
    [T(\vec e_1) | T(\vec e_2) | \cdots | T(\vec e_n)]=
    [a\vec e_1 | a\vec e_2 | \cdots | a\vec e_n] = aI.
  \]
}
\end{enumerate}

\noindent{\bfseries Question 13.} Fix $a \in \R$ and $\vec u \in \R^n$ with $\vec
u \neq \vec 0$. Is the map given by $\vec v \mapsto a\vec v + \vec u$, linear?
Why or why not? \\

\BB{No; it is not a linear map. Let $\vec v_1,\,\vec v_2 \in \R^n$. Then $\vec
  v_1+\vec v_2 \mapsto a\vec v_1+a\vec v_2+\vec u$. But $\vec v_1 \mapsto a\vec
  v_1+\vec u$ and $\vec v_2 \mapsto a\vec v_2+\vec u$. However, $(a\vec v_1+\vec
  u)+(a\vec v_2+\vec u)=a\vec v_1+a\vec v_2+2\vec u \neq a\vec v_1+a\vec
  v_2+\vec u$. \\}

\noindent{\bfseries Question 14.} Consider a linear transformation $T: \R^n
\rightarrow \R^n$, and define $\text{Ker}(T)=\{\vec v \in \R^n : T(\vec v)=\vec
0\}$. This is the kernel of the linear transformation $T$. For $\vec v \in
\R^n$, define $\vec v + \text{Ker}(T)=\{\vec v + \vec u : \vec u \in
\text{Ker}(T)\}$. Show the following.
\begin{enumerate}[(a)]
\item $\text{Ker}(T)$ is closed under scalar multiplcation and vector addition.\\

  \BB{Let $\vec v_1,\,\vec v_2 \in \text{Ker}(T)$. Then $T(\vec v_1+\vec
    v_2)=T(\vec v_1)+T(\vec v_2)=\vec 0+\vec 0=\vec 0$, so $\vec v_1+\vec v_2
    \in \text{Ker}(T)$. Similarly, $T(a\vec v)=aT(\vec v)=a\vec 0=\vec 0$
    whenever $\vec v \in \text{Ker}(T)$; so, $a\vec v \in \text{Ker}(T)$. This
    shows that $\text{Ker}(T)$ is closed under vector addition and scalar
    multiplication. \\}
  
\item For $\vec v \in \R^n$, show that $\vec v + \text{Ker}(T)$ consists of all
  and only those elements of $\R^n$ that map to $T(\vec v)$ under $T$. \\

  \BB{Let $V=\{u \in \R^n : T(\vec u)=T(\vec v)\}$. We show $\vec
    v+\text{Ker}(T)=V$. Indeed, let $\vec u \in V$. Then $T(\vec v)=T(\vec u)$
    so that $T(u-v)=\vec 0$. It follows that there is an $\vec x \in
    \text{Ker}(T)$ such that $\vec u-\vec v=\vec x$, that is, $\vec u=\vec
    v+\vec x$. This means $\vec u \in \vec v+\text{Ker}(T)$.}

  \BB{Conversely, let $\vec u \in \vec v+\text{Ker}(T)$. Then there is some
    $\vec x \in \text{Ker}(T)$ for which $\vec u=\vec v+\vec x.$ It follows that
  $T(\vec u)=T(\vec v+\vec x)+T(\vec v)+T(\vec x)=T(\vec v)+\vec 0=T(\vec v)$.
  So, $\vec u \in V$, as required. \\}

\item For $\vec v_1,\vec v_2 \in \R^n$, show that either $\vec
  v_1+\text{Ker}(T)=\vec v_2 +\text{Ker}(T)$ or $\vec v_1+\text{Ker}(T) \cap
  \vec v_2 +\text{Ker}(T)=\emptyset$. \\

  \BB{Let $\vec v_1,\,\vec v_2 \in \R^n$, and suppose that $\vec
    v_1+\text{Ker}(T) \cap \vec v_2+\text{Ker}(T) \neq \emptyset$. Then there is
    some $\vec u \in v_1+\text{Ker}(T) \cap \vec v_2+\text{Ker}(T)$. By
    definition, there must be $\vec u_1,\,\vec u_2 \in \text{Ker}(T)$ such that
    $\vec u=\vec v_1+\vec u_1=\vec v_2=\vec u_2$. But then $\vec v_1-\vec
    v_2=\vec u_2-\vec u_1=\vec u_3$ for some $\vec u_3 \in \text{Ker}(T)$. It
    follows that $\vec v_1=\vec v_2+\vec u_3$, so $\vec v_1 \in \vec
    v_2+\text{Ker}(T)$. Similarly, $\vec v_2=\vec v_1-\vec u_3$, so $\vec v_2 \in
    \vec v_1+\text{Ker}(T)$. It follows, therefore, that $\vec
    v_1+\text{Ker}(T)=\vec v_2+\text{Ker}(T)$.}
\end{enumerate}

\subsection*{Matrix Operations}

\noindent{\bfseries Question 15.} Give an example of a nonzero matrix $A$ such
that $A^2=O$. \\

\BB{Take $A=\left( \begin{smallmatrix}0&1\\0&0\end{smallmatrix} \right)$. \\}

\noindent{\bfseries Question 16.} The trace of a square matrix $A$ of dimensions
$N \times N$ is defined as $\text{tr}(A)=\sum_{k=1}^NA_{k,k}$, i.e., the sum of
the diagonal entries of the matrix. For any other $N \times N$ matrix $B$, show
that $\text{tr}(AB)=\text{tr}(BA)$. \\

\BB{Observe
  \begin{align*}
    \text{tr}(AB) &= \sum_{k=1}^N(AB)_{k,k} \\
                  &= \sum_{k=1}^N\sum_{j=1}^NA_{k,j}B_{j,k} \\
                  &= \sum_{k=1}^N\sum_{j=1}^NA_{j,k}B_{k,j} \\
                  &= \text{tr}(BA)
  \end{align*}
  where the second to last equality follows because
  \[
    \{(k,j,j,k) : 1 \leqq j,k \leqq N\} =
    \{(j,k,k,j) : 1 \leqq j,k \leqq N\}.
  \] \\
}

\noindent{\bfseries Question 17.} An $N \times N$ matrix $A$ is circulant if it is
of the form
\[
  A=\begin{pmatrix}
      a_1 & a_2 & a_3 & \cdots & a_N \\
      a_N & a_1 & a_2 & \cdots & a_{N-1} \\
      a_{N-1} & a_N & a_1 & \cdots & a_{N-2} \\
      \vdots & \vdots & \vdots & \ddots & \vdots \\
      a_2 & a_3 & a_4 & \cdots & a_1
    \end{pmatrix}.
\]
Show that if $B$ is any other $N \times N$ circulant matrix, then $AB=BA$. \\

\BB{Define the $N \times N$ matrix $G$ by
  \[
    G=\begin{pmatrix}
        0 & 1 & 0 & \cdots & 0 \\
        0 & 0 & 1 & \cdots & 0 \\
        \vdots & \vdots & \vdots & \cdots & \vdots \\
        1 & 0 & 0 & \cdots & 0
      \end{pmatrix}.
  \]
  If we multiply an $N \times N$ matrix $C$ on the right by $G$, the resulting
  matrix is the one obtained by cyclically shifting the columns of $C$ to the
  right. In particular, $G^N=I$ and $G^j \neq I$ for any $j \in
  \{1,\dots,N-1\}$. We also note that we can write $A$ and $B$ by
  \[
    A = \sum_{i=1}^Na_iG^{i-1}, \qquad B = \sum_{i=1}^Nb_iG^{i-1},
  \]
  that is, $A$ and $B$ are polynomials in $G$. Since they are each polynomials
  in $G$, it is easy to see that they must commute. \\}

\noindent{\bfseries Question 18.} A diagonal matrix is one for which every entry
not on the main diagonal is zero. Let $A$ and $B$ be $N \times N$ matrices such
that there exists and invertible $N \times N$ matrix $P$ for which
$D_A=P^{-1}AP$ and $D_B=P^{-1}BP$ are diagonal matrices. Show that $A$ and $B$
commute. \\

\BB{Since $D_A=P^{-1}AP$ and $D_B=P^{-1}BP$, we have that $A=PD_AP^{-1}$ and
  $B=PD_BP^{-1}$. Then
  \begin{align*}
    AB &= (PD_AP^{-1})(PD_BP^{-1}) \\
       &= PD_A(P^{-1}P)D_BP^{-1} \\
       &= PD_AID_BP^{-1} \\
       &= PD_AD_BP^{-1} \\
       &= PD_BD_AP^{-1} \\
       &= PD_BID_AP^{-1} \\
       &= PD_B(P^{-1}P)D_AP^{-1} \\
       &= (PD_BP^{-1})(PD_AP^{-1}) \\
       &= BA
  \end{align*}
  where we have used the fact that diagonal matrices commute.
}

\end{document}