\documentclass[a4paper,11pt]{article}

\usepackage[utf8]{inputenc}
\usepackage[english]{babel}
\usepackage{amssymb, amsmath, amsthm, mathrsfs}
\usepackage[left=1.0in,right=1.0in,top=1.0in,bottom=1.0in]{geometry}
\usepackage[inline,shortlabels]{enumitem}
\usepackage{times}
\usepackage{xcolor}

\newcommand{\R}{\mathbf{R}}
\newcommand{\BB}[1]{\textcolor{blue}{#1}}

\begin{document}

\begin{center}
  {\Large\bfseries Math 240 Tutorial \\ Solutions}
\end{center}
\begin{center}
  {\bfseries June 27}
\end{center}

\noindent{\bfseries Question 1.} Let $T : \R^n \rightarrow \R^n$ be an invertible
linear transformation, and let $\vec v_1,\dots,\,\vec v_m \in \R^n$. Prove
that $\{\vec v_1,\dots,\,\vec v_m\}$ is an independent set if and only if
$\{T(\vec v_1),\dots,\,T(\vec v_m)\}$ is an independent set. \\

% \BB{Suppose $\{\vec v_1,\dots,\,\vec v_m\}$ is an independent set but $\{T(\vec
%   v_1),\dots,\,T(\vec v_m)\}$ is not. Then there are scalars
%   $\alpha_1,\dots,\,\alpha_m \in \R$, not all 0, such that
%   \[
%     \vec 0 = \alpha_1T(\vec v_1) + \cdots + T(\vec v_m) = T(\alpha\vec
%     v_1+\cdots+\alpha_m\vec v_m).
%   \]
%   Since $T$ is invertible, it is 1-to-1. This means
%   \[
%     \vec 0 = \alpha\vec v_1+\cdots+\alpha_m\vec v_m.
%   \]
%   This contradicts the assumed independence of $\{\vec v_1,\dots,\,\vec v_m\}$.
%   So, $\{T(\vec v_1),\dots,\,T(\vec v_m)\}$ is also an independent set.}

% \BB{Conversely, suppose $\{\vec v_1,\dots,\,\vec v_m\}$ is not an independent
%   set. Then there are scalars $\beta_1,\dots,\,\beta_m in \R$, not all 0, such
%   that
%   \[
%     \vec 0 = \alpha\vec v_1+\cdots+\alpha_m\vec v_m.
%   \]
%   Applying $T$ to both sides,
%   \[
%     \vec 0 = \alpha_1T(\vec v_1) + \cdots + T(\vec v_m),
%   \]
%   showing that $\{T(\vec v_1),\dots,\,T(\vec v_m)\}$ is also a dependent set. \\}

\noindent{\bfseries Question 2.} Define $T : \R^n \rightarrow \R^n$ by
\[
  T(x_1,x_2,x_3,x_4) =
  (x_1-x_2-x_3-x_4, -x_1+x_2-x_3-x_4,-x_1-x_2+x_3-x_4,-x_1-x_2-x_3+x_4).
\]
Is $T$ linear? Is $T$ invertible? If it is, what is its inverse? \\

\BB{The components of the image of $\vec x$ are linear combinations of the
  components of $\vec x$, so $T$ is linear. The standard matrix for $T$ is given
  by
  \[
    T_A = 
    \left(
      \begin{array}{rrrr}
        1 & -1 & -1 & -1 \\
        -1 & 1 & -1 & -1 \\
        -1 & -1 & 1 & -1 \\
        -1 & -1 & -1 & 1 \\
      \end{array}
    \right)
  \]
  which is readily verified to be invertible. Moreover, it is its own inverse.
  So $T^{-1}=T$. \\
}

\noindent{\bfseries Question 3.} Show that if $E$ and $F$ are two $n \times n$
matrices such that $EF=I$, then $E$ and $F$ commute.  \\

\BB{Since $EF = I$, $E$ is invertible and $F=E^{-1}$ by the uniqueness of
  $E^{-1}$. Of course, $EE^{-1} = E^{-1}E = I$. \\}

\noindent{\bfseries Question 4.} Let $T : \R^n \rightarrow \R^n$ and $U : \R^n
\rightarrow \R^n$ be two linear transformations such that $T(U(\vec x))=\vec x$
for every $\vec x \in \R^n$. Show that $T$ is invertible and $U=T^{-1}$. \\

\BB{By assumption, $U$ is a right inverse of $T$. Since $T$ is a linear
  operator of a finite dimensional vector space, it is invertible and
  $T^{-1}=U$. \\}

\noindent{\bfseries Question 5.} Show that if $A$ is invertible, then
$\text{det}(A^{-1}) = 1/\text{det}(A)$. \\

\BB{We rewrite the putative equality as $\text{det}(A)\text{det}(A^{-1})=1$. By
  the properties of the determinant,
  \[
    1 = \text{det}(I) = \text{det}(AA^{-1}) = \text{det}(A)\text{det}(A^{-1}),
  \]
  which proves the result. \\}

\noindent{\bfseries Question 6.} Let $A$, $B$, and $P$ be $n \times n$ matrices
where $P$ is invertible and $B = P^{-1}AP$. Show that
$\text{det}(A)=\text{det}(B)$.  \\

\BB{Observe
  \[
    \text{det}(B) = \text{det}(P^{-1}AP) =
    \text{det}(P^{-1})\text{det}(A)\text{det}(P) =
    \frac{\text{det}(A)\text{det}(P)}{\text{det}(P)} = \text{det}(A)
  \]
  where we have used the previous exercise. \\}

\noindent{\bfseries Question 7.} Let $V$ be a vector space, and let $H$ and $K$ be
subspaces of $V$. Show the following
\begin{enumerate}[(a)]
\item $H+K$ and $H \cap K$ are subapces. \\

  \BB{Let $\vec v_1,\,\vec v_2 \in H+K$. Then there are vectors $\vec h_1,\,\vec
    h_2 \in H$ and $\vec k_1,\,\vec k_2 \in K$ such that $\vec v_1=\vec h_1+\vec
    k_1$ and $\vec v_2=\vec h_2+\vec k_2$. Then $\vec v_1 + \vec v_2 = (\vec
    h_1+\vec h_2)+(\vec k_1+\vec k_2)$. Since $\vec h_1+\vec h_2 \in H$ and
    $\vec k_1+\vec k_2 \in K$, we have $\vec v_1 + \vec v_2 \in H+K$. If $\alpha
    \in \R$, then $\alpha\vec v = \alpha\vec h + \alpha\vec k$. Since
    $\alpha\vec h \in H$ and $\alpha\vec k \in K$, $\alpha\vec v \in H+K$.
    Finally, $H+K \neq \emptyset$ since $\vec 0 \in H+K$. We have therefore
    shown that $H+K$ is a subspace of $V.$}

  \BB{Let $\vec v_1,\,\vec v_2 \in H \cap K$. Then $\vec v_1,\,\vec v_2 \in H$
    and $\vec v_1,\,\vec v_2 \in K$. Therefore, $\vec v_1+\vec v_2 \in H$ and
    $\vec v_1+\vec v_2 \in K$ so that $\vec v_1+\vec v_2 \in H \cap K$.
    Similarly, for $\alpha \in \R$, $\alpha\vec v \in H$ and $\alpha\vec v \in
    K$ so that $\alpha\vec v \in H \cap K$. Since $\vec 0 \in H$ and $\vec 0 \in
    K$, we have $\vec 0 \in H \cap K$ so that $H \cap K \neq \emptyset$. We have
    shown that $H \cap K$ is a subsapce of $V$. \\}
  
\item $H$ and $K$ are subspaces of $H+K$. \\

  \BB{Note that $\vec 0 \in H$ and $\vec 0 \in K$. So, for $\vec h \in H$ and
    $\vec k \in K$, we have $\vec h + \vec 0 \in H+K$ and $\vec 0 +\vec k \in
    H+K$. This shows $H \subseteqq H+K$ and $K \subseteqq H+K$. \\}
  
\item $H \cap K$ is a subspace of both $H$ and $K$. \\

  \BB{We have already verified $H \cap K$ to be a subspace of the ambient space
    $V$, and clearly $H \cap K \subseteqq H$ and $H \cap K \subseteqq K$,
    whereupon $H \cap K$ is a linear subspace of both $H$ and $K$. \\}
\end{enumerate}

\noindent{\bfseries Question 8.} Let $V$ be a vector space, and let $W$ be a
vector space of $V$. Recall that, for $\vec v \in V$, $\vec v + W = \{\vec
v+\vec w : \vec w \in W\}$. Show the following.
\begin{enumerate}[(a)]
\item For distinct $\vec v_1,\,\vec v_2 \in V$, $\vec v_1+W$ and $\vec v_2+W$
   are either disjoint or equal. \\

  \BB{Assume $\vec v_1 + W \cap \vec v_2 + W \neq \emptyset$, and let $\vec x
    \in \vec v_1 + W \cap \vec v_2 + W$. Then there are $\vec w_1,\,\vec w_2 \in
    W$ such that $\vec x = \vec v_1+\vec w_1=\vec v_2+\vec w_2$, whereupon $\vec
    v_1 = \vec v_2+\vec w_3$ where $\vec w_3=\vec w_2-\vec w_1 \in W$. Let $\vec
    v_1+\vec w$ be an arbitary element of $\vec v_1+W$. Then $\vec v_1+\vec w =
    \vec v_2+(\vec w_3+\vec w)$ which shows that $\vec v_1+W \subseteqq \vec
    v_2+W$. Similarly, $\vec v_2+W \subseteqq \vec v_1+W$ so that $\vec
    v_1+W=\vec v_2+W$, as desired. \\}
  
\item $\vec v_1+W = \vec v_2+W$ if and only if $\vec v_1 - \vec v_2 \in W$. \\

  \BB{Assume $\vec v_1+W = \vec v_2+W$. In particular, $\vec v_1 \in \vec
    v_2+W$ so that there is a $\vec w \in W$ for which $\vec v_1=\vec v_2+\vec
    w$; in other words, $\vec v_1-\vec v_2 = \vec w \in W$.}

  \BB{Conversely, suppose that $\vec v_1 - \vec v_2 = \vec w$ for some $\vec w
    \in W$. Then $\vec v_1 = \vec v_2 + \vec w$ so that $\vec v_1 \in \vec
    v_2+W$. If $\vec v_1+\vec w' \in \vec v_1+W$ is arbitrary, then $\vec
    v_1+\vec w' = \vec v_2+(\vec w+\vec w') \in \vec v_2+W$ since $\vec w+\vec
    w' \in W$. Therefore, $\vec v_1+W \subseteqq \vec v_2+W$. Similarly, $\vec
    v_2+W \subseteqq \vec v_1+W$, hence $\vec v_1+W = \vec v_2+W$. \\}
  
\item Every $\vec v \in V$ belongs to $\vec u+W$ for some $\vec u \in V$. \\

  \BB{Since $\vec 0 \in W$, it follows that $\vec v = \vec v+\vec 0 \in \vec v
    +W$. \\}
\end{enumerate}
We can define an arithmetic on $H = \{\vec v+W : \vec v \in V\}$. For $\vec
v_1+W,\,\vec v_2+W \in H$ and $\alpha \in \R$, define $(\vec v_1+W)+(\vec v_2+W)
= (\vec v_1+\vec v_2)+W$ and $\alpha(\vec v_1 + W) = (\alpha\vec v_1)+W$. Then:
\begin{enumerate}[(a)]
  \setcounter{enumi}{3}
\item $H$ is a vector under the arithmetic defined above. We call $H$ the
  quotient space of $V$ by $W$, and we denote it as $H = V/W$.
\end{enumerate}

\end{document}