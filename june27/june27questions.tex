\documentclass[a4paper,11pt]{article}

\usepackage[utf8]{inputenc}
\usepackage[english]{babel}
\usepackage{amssymb, amsmath, amsthm, mathrsfs}
\usepackage[left=1.0in,right=1.0in,top=1.0in,bottom=1.0in]{geometry}
\usepackage[inline,shortlabels]{enumitem}
\usepackage{times}
\usepackage{xcolor}

\newcommand{\R}{\mathbf{R}}
\newcommand{\PP}{\mathbf{P}}

\begin{document}

\begin{center}
  {\Large\bfseries Math 240 Tutorial \\ Questions}
\end{center}
\begin{center}
  {\bfseries June 27}
\end{center}

\noindent{\bfseries Question 1.} Let $T : \R^n \rightarrow \R^n$ be an invertible
linear transformation, and let $\vec v_1,\dots,\,\vec v_m \in \R^n$. Prove
that $\{\vec v_1,\dots,\,\vec v_m\}$ is an independent set if and only if
$\{T(\vec v_1),\dots,\,T(\vec v_m)\}$ is an independent set. \\

\noindent{\bfseries Question 2.} Define $T : \R^n \rightarrow \R^n$ by
\[
  T(x_1,x_2,x_3,x_4) =
  T(x_1-x_2-x_3-x_4, -x_1+x_2-x_3-x_4,-x_1-x_2+x_3-x_4,-x_1-x_2-x_3+x_4).
\]
Is $T$ linear? Is $T$ invertible? If it is, what is its inverse? \\

\noindent{\bfseries Question 3.} Show that if $E$ and $F$ are two $n \times n$
matrices such that $EF=I$, then $E$ and $F$ commute.  \\

\noindent{\bfseries Question 4.} Let $T : \R^n \rightarrow \R^n$ and $U : \R^n
\rightarrow \R^n$ be two linear transformations such that $T(U(\vec x))=\vec x$
for every $\vec x \in \R^n$. Show that $T$ is invertible and $U=T^{-1}$. \\

\noindent{\bfseries Question 5.} Show that if $A$ is invertible, then
$\text{det}(A^{-1}) = 1/\text{det}(A)$. \\

\noindent{\bfseries Question 6.} Let $A$, $B$, and $P$ be $n \times n$ matrices
where $P$ is invertible and $B = P^{-1}AP$. Show that
$\text{det}(A)=\text{det}(B)$.  \\

\noindent{\bfseries Question 7.} Let $V$ be a vector space, and let $H$ and $K$ be
subspaces of $V$. Show the following
\begin{enumerate}[(a)]
\item $H+K$ and $H \cap K$ are subapces.
\item $H$ and $K$ are subspaces of $H+K$.
\item $H \cap K$ is a subspace of both $H$ and $K$. \\
\end{enumerate}

\noindent{\bfseries Question 8.} Let $V$ be a vector space, and let $W$ be a
vector space of $V$. Recall that, for $\vec v \in V$, $\vec v + W = \{\vec
v+\vec w : \vec w \in W\}$. Show the following.
\begin{enumerate}[(a)]
\item For distinct $\vec v_1,\,\vec v_2 \in V$, $\vec v_1+W$ and $\vec v_2+W$
  are either disjoint or equal.
\item $\vec v_1+W = \vec v_2+W$ if and only if $\vec v_1 - \vec v_2 \in W$.
\item Every $\vec v \in V$ belongs to $\vec u+W$ for some $\vec u \in V$.
\end{enumerate}
We can define an arithmetic on $H = \{\vec v+W : \vec v \in V\}$. For $\vec
v_1+W,\,\vec v_2+W \in H$ and $\alpha \in \R$, define $(\vec v_1+W)+(\vec v_2+W)
= (\vec v_1+\vec v_2)+W$ and $\alpha(\vec v_1 + W) = (\alpha\vec v_1)+W$. Then:
\begin{enumerate}[(a)]
  \setcounter{enumi}{3}
\item $H$ is a vector under the arithmetic defined above. We call $H$ the
  quotient space of $V$ by $W$, and we denote it as $H = V/W$.
\end{enumerate}

\end{document}