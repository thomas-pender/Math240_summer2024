\documentclass[a4paper,11pt]{article}

\usepackage[utf8]{inputenc}
\usepackage[english]{babel}
\usepackage{amssymb, amsmath, amsthm, mathrsfs}
\usepackage[left=1.0in,right=1.0in,top=1.0in,bottom=1.0in]{geometry}
\usepackage[inline,shortlabels]{enumitem}
\usepackage{times}
\usepackage{xcolor}

\newcommand{\R}{\mathbf{R}}

\begin{document}

\begin{center}
  {\Large\bfseries Math 240 Tutorial \\ Questions}
\end{center}
\begin{center}
  {\bfseries May 30}
\end{center}

\noindent{\bfseries Question 1.} Consider the set
$S=\{(1,3,-4,2),(2,2,-4,0),(1,-3,2,-4),(-1,0,1,0)\}$ of vectors in $\R^4$. Show
they form a linearly dependent set, and exress on vector as a linear combination
of the others. \\

\noindent{\bfseries Question 2.} Consider the set
$S=\{(1,0,0,-1),(0,1,0,-1),(0,0,1,-1),(0,0,0,1)\}$ of vectors in $\R^4$. Show
they form a linearly independent set. For a general vector $(a_1,a_2,a_3,a_4)
\in \R^4$, derive the coefficients for this vector when it is expanded as a
linear combination of the vectors in $S$. \\

\noindent{\bfseries Question 3.} Let $S_1$ and $S_2$ be finite subsets of $\R^n$,
for some $n$, such that $S_1 \subseteqq S_2$. Prove that if $S_1$ is a linearly
dependent set, then so is $S_2$. Show that this is equivalent to if $S_2$ is a
linearly independent set, then so is $S_1$. \\

\noindent{\bfseries Question 4.} Let $S$ be a linearly independent set of $\R^n$,
and let $\vec v$ be a vector in $\R^n$ that is not in $S$. Prove that $S \cup
\{\vec v\}$ is linearly dependent if and only if $\vec v \in \text{span}(S)$. \\

\noindent{\bfseries Question 5.} Do the following.
\begin{enumerate}[(a)]
\item Let $\vec u$ and $\vec v$ be distinct vectors in $\R^n$. Prove that
  $\{\vec u, \vec v\}$ is linear independent if and only if $\{\vec u + \vec v,
  \vec u - \vec v\}$ is linearly independent.
\item Let $\vec u, \vec v, \vec w$ be distinct vectors in $\R^n$. Prove that
  $\{\vec u, \vec v, \vec w\}$ is linearly independent if and only if $\{\vec
  u+\vec v,\vec u+\vec w,\vec v+\vec w\}$ is linear independent. \\
\end{enumerate}

\noindent{\bfseries Question 6.} Show the following for $\R^n$.
\begin{enumerate}[(a)]
\item Show that scalar multiplication is a linear transformation.
\item When is this linear map invertible?
\item Is its inverse a linear transformation?
\item Fix an element $a \in \R^n$. What is the matrix corresponding to the
linear transformation $\vec v \mapsto a\vec v$? \\
\end{enumerate}

\noindent{\bfseries Question 7.} Fix $a \in \R$ and $\vec u \in \R^n$. Is the map
given by $\vec v \mapsto a\vec v + \vec u$ linear? Why or why not? \\

\noindent{\bfseries Question 8.} Consider a linear transformation $T: \R^n
\rightarrow \R^n$, and define $\text{Ker}(T)=\{\vec v \in \R^n : T(\vec v)=\vec
0\}$. This is the kernel of the linear transformation $T$. For $\vec v \in
\R^n$, define $\vec v + \text{Ker}(T)=\{\vec v + \vec u : \vec u \in
\text{Ker}(T)\}$. Show the following.
\begin{enumerate}[(a)]
\item $\text{Ker}(T)$ is closed under scalar multiplcation and vector addition.
\item For $\vec v \in \R^n$, show that $\vec v + \text{Ker}(T)$ consists of all
  and only those elements of $\R^n$ that map to $\vec v$ under $T$.
\item For $\vec v_1,\vec v_2 \in \R^n$, show that either $\vec
  v_1+\text{Ker}(T)=\vec v_2 +\text{Ker}(T)$ or $\vec v_1+\text{Ker}(T) \cap
  \vec v_2 +\text{Ker}(T)=\emptyset$.
\end{enumerate}

\end{document}