\documentclass[a4paper,11pt]{article}

\usepackage[utf8]{inputenc}
\usepackage[english]{babel}
\usepackage{amssymb, amsmath, amsthm, mathrsfs}
\usepackage[left=1.0in,right=1.0in,top=1.0in,bottom=1.0in]{geometry}
\usepackage[inline,shortlabels]{enumitem}
\usepackage{times}
\usepackage{xcolor}

\newcommand{\R}{\mathbf{R}}
\newcommand{\PP}{\mathbf{P}}
\newcommand{\ddim}{\text{dim}}

\begin{document}

\begin{center}
  {\Large\bfseries Math 240 Tutorial \\ Questions}
\end{center}
\begin{center}
  {\bfseries July 5}
\end{center}

\noindent{\bfseries Question 1.} Consider the vector space $\R^3$, and let
\[
  H=\left\{
    \begin{pmatrix}a\\0\\0\end{pmatrix}
    : a \in \R
  \right\}.
\]
Answer the following.
\begin{enumerate}[(a)]
\item Show that $H$ is a subspace of $\R^3$.
\item What is the dimension of $H$?
\item Construct a basis for $H$. \\
\end{enumerate}

\noindent{\bfseries Question 2.} Let $\PP_3$ be the vector space of all polynomials
of degree at most 3, and let
\[
  H = \{p(x) \in \PP_3 : p(3)=0\}.
\]
Answer the following.
\begin{enumerate}[(a)]
\item Show that $H$ is a subspace of $\PP_3$.
\item What is the dimension of $H$?
\item Construct a basis for $H$.
\item Let $\PP_2$ be the vector space of polynomials of degree at most $2$.
  $\PP_2$ is a subspace of $\PP_3$ (why?). Give an invertible linear
  transformation that maps $\PP_2$ onto $H$. What is the matrix for the
  transformation with respect to the standard basis of $\PP_3$? \\
\end{enumerate}

\noindent{\bfseries Question 3.} Let $m$ and $n$ be positive integers. Show the
following.
\begin{enumerate}[(a)]
\item The set $M_{m \times n}(\R)$ of $m \times n$ matrices with real entries is
  a vector space.
\item What is the dimension of $M_{m \times n}(\R)$?
\item Construct a basis for $M_{m \times n}(\R)$.
\item Show the subset of matrices with trace 0 forms a subspace of $M_{n \times
    n}(\R)$. Use the Dimension Theorem to find the dimension of this subset. You
  will first need to show that the trace map is a linear map. Construct a basis
  for this subspace. \\
\end{enumerate}

\noindent{\bfseries Question 4.} Define
\begin{align*}
  H &=
      \left\{
      \begin{pmatrix}
        u & -u-x \\ 0 & x
      \end{pmatrix}
      : u,x \in \R
      \right\}, \\
  K &=
      \left\{
      \begin{pmatrix}
        v & 0 \\ w & -v
      \end{pmatrix}
      \right\}.
\end{align*}
Do the following.
\begin{enumerate}[(a)]
\item $H$ and $K$ are subspaces of $M_{2 \times 2}(\R)$.
\item Construct bases for $H$, $K$, $H+K$, and $H \cap K$. \\
\end{enumerate}

\noindent{\bfseries Question 5.} Your course text proves the Dimension Theorem by
counting pivot positions in matrices. Prove the theorem by arguing from the
general definitions, without recourse to matrices, in the following way. Given a
linear transformation $T : V \rightarrow W$, take a basis for $\text{ker}(T)$
and enlarge it to a basis for $V$. Apply $T$ to the vectors that were added to
the basis of $\text{ker}(T)$, and argue that they form a linearly independent
set that spans the range of $T$ in $W$. \\

\noindent{\bfseries Question 6.} Use the Dimension Theorem to show a linear
transformation $T: V \rightarrow V$ is invertible if and only if it is onto if
and only if it is one-to-one. \\

\noindent{\bfseries Question 7.} Let $V$ be a vector space of finite dimension,
and let $H$ and $K$ be subspaces of $V$. Show
\[
  \ddim(H+K) = \ddim(H) + \ddim(K) - \ddim(H \cap K).
\]

\end{document}