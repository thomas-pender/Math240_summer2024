\documentclass[a4paper,11pt]{article}

\usepackage[utf8]{inputenc}
\usepackage[english]{babel}
\usepackage{amssymb, amsmath, amsthm, mathrsfs}
\usepackage[left=1.0in,right=1.0in,top=1.0in,bottom=1.0in]{geometry}
\usepackage[inline,shortlabels]{enumitem}
\usepackage{times}
\usepackage{xcolor}

\newcommand{\R}{\mathbf{R}}
\newcommand{\C}{\mathbf{C}}
\newcommand{\M}{\mathcal{M}}
\newcommand{\PP}{\mathbf{P}}
\newcommand{\ddim}{\text{dim}}
\newcommand{\blue}[1]{\textcolor{blue}{#1}}

\begin{document}

\begin{center}
  {\Large\bfseries Math 240 Tutorial \\ Solutions}
\end{center}
\begin{center}
  {\bfseries August 1}
\end{center}

\noindent{\bfseries Question 1.} Find the unit vector in the direction of the given
vectors
\begin{enumerate*}[(a)]
\item $\left( \begin{smallmatrix}-30\\40\end{smallmatrix} \right)$,
\item $\left( \begin{smallmatrix}7/4\\1/2\\1\end{smallmatrix} \right)$, and
\item $\left( \begin{smallmatrix}8/3\\2\end{smallmatrix} \right)$. \\
\end{enumerate*}

\blue{
  \begin{enumerate}[(a)]
  \item We have $\Vert \vec v \Vert = \sqrt{\langle \vec v,\,\vec v \rangle} =
    \sqrt{(-30)^2+40^2} = 50$ so that the unit vector is given by $\frac{\vec
      v}{\Vert \vec v
      \Vert}=\left( \begin{smallmatrix}-3/5\\4/5\end{smallmatrix} \right)$.
  \item We have $\Vert \vec v \Vert = \sqrt{\langle \vec v,\,\vec v \rangle} =
    \sqrt{(7/4)^2+(1/2)^2+1^2} = \frac{1}{2}\sqrt{49+4+16}=\frac{\sqrt{69}}{2}$
    so that the unit vector is given by $\frac{\vec v}{\Vert \vec v
      \Vert}=\frac{1}{\sqrt{69}}\left(
      \begin{smallmatrix}7/2\\1\\2\end{smallmatrix}\right)$.
  \item We have $\Vert v \Vert = \sqrt{\langle \vec v,\,\vec v \rangle} =
    \sqrt{(8/3)^2+2^2} = \frac{10}{3}$ so that the unit vector is given by
    $\frac{\vec v}{\Vert \vec v
      \Vert}=\left( \begin{smallmatrix}4/5\\3/5\end{smallmatrix} \right)$. \\
  \end{enumerate}
}

\noindent{\bfseries Question 2.}
\begin{enumerate*}[(a)]
\item Let $\vec u_1=\left( \begin{smallmatrix}2\\-3\end{smallmatrix} \right)$,
  $\vec u_2=\left( \begin{smallmatrix}6\\4\end{smallmatrix} \right)$, and $\vec
  x=\left( \begin{smallmatrix}9\\-7\end{smallmatrix} \right)$. Does $\{\vec
  u_1,\,\vec u_2\}$ form an orthogonal basis for $\R^2$? If it does, write $\vec
  x$ in terms of this basis.
\item Compute the orthogonal projection of
  $\vec x=\left( \begin{smallmatrix}1\\7\end{smallmatrix} \right)$ onto the line
  through $\vec y = \left( \begin{smallmatrix}-4\\2\end{smallmatrix} \right)$
  and the origin. \\
\end{enumerate*}

\blue{
  \begin{enumerate}[(a)]
  \item Note that $\langle \vec u_1,\,\vec u_2 \rangle = 12-12 = 0$ so that
    $\{\vec u_1,\,\vec u_2\}$ is an orthogonal basis for $\R^2$. We have
    \[
      \vec x = \frac{\langle \vec x,\,\vec u_1 \rangle}{\Vert \vec u_1
        \Vert}\vec u_1 + \frac{\langle \vec x,\,\vec u_2 \rangle}{\Vert \vec u_2
        \Vert}\vec u_2 = \frac{39}{\sqrt{13}}\vec u_1 + \frac{13}{\sqrt{13}}\vec
      u_2.
    \]
  \item The projection is given by
    \[
      \frac{\langle \vec x,\,\vec y \rangle}{\Vert \vec y \Vert}\vec y
      = \frac{-4+14}{\sqrt{16+4}}
      \left(
        \begin{array}{r}
          -4\\2
        \end{array}
      \right)
      =\frac{5}{\sqrt{5}}
      \left(
        \begin{array}{r}
          -2\\1
        \end{array}
      \right).
    \] \\
  \end{enumerate}
}

\noindent{\bfseries Question 3.} Let $\vec y \in \R^n$. Prove $\vec x \mapsto
\langle \vec x,\,\vec y \rangle$ is a linear transformation $\R^n \rightarrow
\R$. \\

\blue{Let $\vec w,\,\vec x \in \R^n$, and let $\alpha \in \R$. Then
  \[
    \langle \vec w+\vec x,\,\vec y \rangle =
    \sum_{j=1}^n (w_j+x_j)y_j = \sum_{j=1}^nw_jy_j+\sum_{j=1}^nx_jy_j = \langle
    \vec w,\,\vec y \rangle + \langle \vec x,\,\vec y \rangle.
  \]
  and
  \[
    \langle \alpha\vec w,\,\vec y \rangle = \sum_{j=1}^n\alpha w_jy_j =
    \alpha\sum_{j=1}^nw_jy_j = \alpha\langle \vec w,\,\vec y \rangle.
  \]
  It follows, therefore, that the map is a linear map. \\}

\noindent{\bfseries Question 4.} Let
\[
  \vec y =
  \left(
    \begin{array}{r}
      -1\\4\\3
    \end{array}
  \right), \quad
  \vec u_1 =
  \left(
    \begin{array}{r}
      1\\1\\0
    \end{array}
  \right), \quad
  \vec u_2 =
  \left(
    \begin{array}{r}
      -1\\1\\0
    \end{array}
  \right).
\]

\noindent Verify that $\{\vec u_1,\,\vec u_2\}$ is an orthogonal set, and find
the orthogonal projection of $\vec y$ onto $\text{span}\{\vec u_1,\,\vec u_2\}$.
Construct a nonzero vector $\vec z$ that is orthogonal to $\vec u_1$ and $\vec
u_2$. Find the distance from $\vec y$ to $\text{span}\{\vec u_1,\,\vec u_2\}$.
\\

\blue{Observe $\langle \vec u_1,\,\vec u_2 \rangle=-1+1=0$, so they are
  orthogonal. To simplify the calculations, however, observe that
  $\text{span}\{\vec u_1,\,\vec u_2\} = \text{span}\{\hat u_1,\,\hat u_2'\}$
  where
  \[
    \hat u_1 =
    \left(
      \begin{array}{r}
        1 \\ 0 \\ 0
      \end{array}
    \right), \qquad
    \hat u_2 =
    \left(
      \begin{array}{r}
        0 \\ 1 \\ 0
      \end{array}
    \right).
  \]
  Applying Gram--Schmidt with this new basis gives that the projection is given
  by
  \[
    \hat y = \langle y,\,\hat u_1 \rangle\hat u_1 + \langle y,\,\hat u_2
    \rangle\hat u_2 = -\hat u_1 + 4\hat u_2 =
    \left(
      \begin{array}{r}
        -1\\4\\0
      \end{array}
    \right).
  \]
  A vector orthogonal to the span is given by
  \[
    \vec z = \vec y-\hat y =
    \left(
      \begin{array}{r}
        0\\0\\3
      \end{array}
    \right).
  \]
  Therefore, the distance from $y$ to $\text{span}\{\vec u_1,\,\vec u_2\}$ is
  given by $\Vert \vec z \Vert = 3$. \\}

\noindent{\bfseries Question 5.} Let $W$ be a subspace of $\R^n$ with an
orthogonal basis $\beta_1=\{\vec w_1, \dots,\,\vec w_p\}$, and let
$\beta_2=\{\vec v_1,\dots,\,\vec v_q\}$ be an orthogonal basis for $W^\perp$.
\begin{enumerate}[(a)]
\item Explain why $\beta_1 \cup \beta_2$ is an orthogonal set. \\

  \blue{Every vector in $\beta_1$ is orthogonal to every other vector in
    $\beta_1$ as well as every vector in $\beta_2$. Similarly, every vector in
    $\beta_2$ is orthogonal to every other vector in $\beta_2$ as well as every
    vector in $\beta_1$. \\}

\item Explain why the set in part (a) spans $\R^n$. \\

  \blue{Every vector $\vec x$ in $\R^n$ can be written uniquely as $\vec x =
    \vec w_1 + \vec w_2$ where $\vec w_1 \in W$ and $\vec w_2 \in W^\perp$. \\}

\item Show that $\dim(W)+\dim(W)=n$. \\

  \blue{Part (b) shows that $\R^n = W \cup W^\perp$. Recall that
    $n=\dim(\R^n)=\dim(W \cup W^\perp) = \dim(W) + \dim(W^\perp) - \dim(W \cap
    W^\perp)$. But $W \cap W^\perp = \{0\}$ has dimension 0, so
    $\dim(W)+\dim(W^\perp)=n$. \\}
\end{enumerate}

\noindent{\bfseries Question 6.} Let $A$ be an $m \times n$ matrix with linearly
independent columns, and let $A=QR$ be its $QR$-factorization. Prove that $R$ is
invertible with positive eigenvalues. \\

\blue{Note that if $R\vec x=\vec 0$, then $A \vec x = QR\vec x = Q\vec 0 = \vec
  0$. Since $A$ has linearly independent columns, it must be that $\vec x\vec
  0$, and it follows that $R$ is invertible.}

\blue{The proof in the text shows that $R$ is an upper triangular matrix with
  nonnegative diagonal entries. Since $R$ is invertible, these entries are
  actually positive. The eigenvalues of a triangular matrix are its diagnal
  entries. \\}

\noindent{\bfseries Question 7.} Recall that
$H=\text{span}\{x-3,\,x^2-3x\}$ is the subspace of $\PP_2(\R)$
consisting of all those vectors divisible by $x-3$. Do the following.
\begin{enumerate}[(a)]
\item Verify that $\langle p,q \rangle \equiv \int_{-1}^1pq\text{ d}x$ is an
  inner product on $\PP_n(\R)$. \\

  \blue{This simply follows by the linearity of the integral. \\}

\item Use the Gram--Schmidt Process to find an orthogonal basis $\beta$ for $H$.
  \\

  \blue{We construct the orthogonal basis $\beta=\{\vec f_1,\,\vec f_2\}$.
    First, take $\vec f_1=x-3$. Then we calculate
    \[
      \vec f_2 = x^2-3x - \frac{\langle x^2-3x,\,\vec f_1 \rangle}{\Vert \vec f_1 \Vert}\vec f_1 = x^2-\left( 3-\frac{6}{\sqrt{14}} \right)x-\frac{9}{\sqrt{14}}.
    \] \\
  }

\item Let $T$ be the linear map $H \rightarrow \PP_1(\R)$ defined by $p
  \mapsto \frac{\text{d}p}{\text{d}x}$. Find the $QR$-factorization of
  $[T]_{\beta}^\gamma$ where $\gamma$ is the standard basis $\{1,\,x\}$ of
  $\PP_1(\R)$. \\

  \blue{We have
    \[
      [T]_\beta^\gamma = \left[ \left[ \frac{\text{d}f_1}{\text{d}x} \right]_S
        \mid \left[ \frac{\text{d}f_2}{\text{d}x} \right] \right] =
      \begin{pmatrix}
        1 & -3+\frac{6}{\sqrt{14}} \\
        0 & 2
      \end{pmatrix}.
    \]
    But this is already in the required form. \\}
\end{enumerate}

\end{document}
