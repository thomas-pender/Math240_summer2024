\documentclass[a4paper,11pt]{article}

\usepackage[utf8]{inputenc}
\usepackage[english]{babel}
\usepackage{amssymb, amsmath, amsthm, mathrsfs}
\usepackage[left=1.0in,right=1.0in,top=1.0in,bottom=1.0in]{geometry}
\usepackage[inline,shortlabels]{enumitem}
\usepackage{times}
\usepackage{xcolor}

\newcommand{\R}{\mathbf{R}}
\newcommand{\C}{\mathbf{C}}
\newcommand{\M}{\mathcal{M}}
\newcommand{\PP}{\mathbf{P}}
\newcommand{\ddim}{\text{dim}}

\begin{document}

\begin{center}
  {\Large\bfseries Math 240 Tutorial \\ Questions}
\end{center}
\begin{center}
  {\bfseries August 1}
\end{center}

\noindent{\bfseries Question 1.} Find the unit vector in the direction of the given
vectors
\begin{enumerate*}[(a)]
\item $\left( \begin{smallmatrix}-30\\40\end{smallmatrix} \right)$,
\item $\left( \begin{smallmatrix}7/4\\1/2\\1\end{smallmatrix} \right)$, and
\item $\left( \begin{smallmatrix}8/3\\2\end{smallmatrix} \right)$. \\
\end{enumerate*}

\noindent{\bfseries Question 2.}
\begin{enumerate*}[(a)]
\item Let $\vec u_1=\left( \begin{smallmatrix}2\\-3\end{smallmatrix} \right)$,
  $\vec u_2=\left( \begin{smallmatrix}6\\4\end{smallmatrix} \right)$, and $\vec
  x=\left( \begin{smallmatrix}9\\-7\end{smallmatrix} \right)$. Does $\{\vec
  u_1,\,\vec u_2\}$ form an orthogonal basis for $\R^2$? If it does, write $\vec
  x$ in terms of this basis.
\item Compute the orthogonal projection of
  $\left( \begin{smallmatrix}1\\7\end{smallmatrix} \right)$ onto the line
  through $\left( \begin{smallmatrix}-4\\2\end{smallmatrix} \right)$ and the
  origin. \\
\end{enumerate*}

\noindent{\bfseries Question 3.} Let $\vec y \in \R^n$. Prove $\vec x \mapsto
\langle \vec x,\,\vec y \rangle$ is a linear transformation $\R^n \rightarrow
\R$. \\

\noindent{\bfseries Question 4.} Let
\[
  \vec y =
  \left(
    \begin{array}{r}
      -1\\4\\3
    \end{array}
  \right), \quad
  \vec u_1 =
  \left(
    \begin{array}{r}
      1\\1\\0
    \end{array}
  \right), \quad
  \vec u_2 =
  \left(
    \begin{array}{r}
      -1\\1\\0
    \end{array}
  \right).
\]

\noindent Verify that $\{\vec u_1,\,\vec u_2\}$ is an orthogonal set, and find
the orthogonal projection of $\vec y$ onto $\text{span}\{\vec u_1,\,\vec u_2\}$.
Construct a nonzero vector $\vec z$ that is orthogonal to $\vec u_1$ and $\vec
u_2$. Find the distance from $\vec y$ to $\text{span}\{\vec u_1,\,\vec u_2\}$.
\\

\noindent{\bfseries Question 5.} Let $W$ be a subspace of $\R^n$ with an orthogonal
basis $\beta_1=\{\vec w_1, \dots,\,\vec w_p\}$, and let $\beta_2=\{\vec
v_1,\dots,\,\vec v_q\}$ be an orthogonal basis for $W^\perp$.
\begin{enumerate}[(a)]
\item Explain why $\beta_1 \cup \beta_2$ is an orthogonal set.
\item Explain why the set in part (a) spans $\R^n$.
\item Show that $\dim(W)+\dim(W)=n$. \\
\end{enumerate}

\noindent{\bfseries Question 6.} Let $A$ be an $m \times n$ matrix with linearly
independent columns, and let $A=QR$ be its $QR$-factorization. Prove that $R$ is
invertible with positive eigenvalues. \\

\noindent{\bfseries Question 7.} Do the following.
\begin{enumerate}[(a)]
\item Verify that $\langle p,q \rangle \equiv \int_{-1}^1pq\text{ d}x$ is an
  inner product on $\PP_n(\R)$.
\item Recall the standard basis $\beta=\{1,\,x,\,x^2\}$ for $\PP_2(\R)$. Use the
  Gram--Schmidt Process to find an orthogonal basis $\beta'$ for $\PP_2(\R)$.
\item Let $T$ be the linear map $\PP_2(\R) \rightarrow \PP_1(\R)$ defined by $p
  \mapsto \frac{\text{d}p}{\text{d}x}$. Find the $QR$-factorization of
  $[T]_\beta$.
\end{enumerate}

\end{document}
