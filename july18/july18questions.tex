\documentclass[a4paper,11pt]{article}

\usepackage[utf8]{inputenc}
\usepackage[english]{babel}
\usepackage{amssymb, amsmath, amsthm, mathrsfs}
\usepackage[left=1.0in,right=1.0in,top=1.0in,bottom=1.0in]{geometry}
\usepackage[inline,shortlabels]{enumitem}
\usepackage{times}
\usepackage{xcolor}

\newcommand{\R}{\mathbf{R}}
\newcommand{\C}{\mathbf{C}}
\newcommand{\PP}{\mathbf{P}}
\newcommand{\ddim}{\text{dim}}

\begin{document}

\begin{center}
  {\Large\bfseries Math 240 Tutorial \\ Questions}
\end{center}
\begin{center}
  {\bfseries July 18}
\end{center}

\noindent{\bfseries Question 1.} For the following matrices, give a basis for their
null space.
\begin{enumerate}[(a)]
\item
  \[
    A =
    \left(
      \begin{array}{rrrr}
        1 & 3 & 5 & 0 \\ 0 & 1 & 4 & -2
      \end{array}
    \right).
  \]
\item
  \[
    A =
    \left(
      \begin{array}{rrrr}
        1 & -6 & 4 & 0 \\ 0 & 0 & 2 & 0
      \end{array}
    \right).
  \]
\item
  \[
    A =
    \left(
      \begin{array}{rrrrr}
        1 & -2 & 0 & 4 & 0 \\
        0 & 0 & 1 & -9 & 0 \\
        0 & 0 & 0 & 0 & 1
      \end{array}
    \right).
  \]
\end{enumerate}

\noindent{\bfseries Question 2.} Find a basis for the space spanned by
\[
  \left(
    \begin{array}{r}
      -8\\7\\6\\5\\-7
    \end{array}
  \right),
  \left(
    \begin{array}{r}
      8\\-7\\-9\\-5\\7
    \end{array}
  \right),
  \left(
    \begin{array}{r}
      -8\\7\\4\\5\\-7
    \end{array}
  \right),
  \left(
    \begin{array}{r}
      1\\4\\9\\6\\-7
    \end{array}
  \right),
  \left(
    \begin{array}{r}
      -9\\3\\-4\\-1\\0
    \end{array}
  \right).
\] \\

\noindent{\bfseries Question 3.} Given vectors $\vec u_1, \dots,\, \vec u_p$ in a
vector space $V$, show $\vec x$ is a linear combination of $\vec u_1, \dots,\,
\vec u_p$ if and only if $[\vec x]_B$ is a linear combination of $[\vec u_1]_B,
\dots,\, [\vec u_p]_B$. \\

\noindent{\bfseries Question 4.} Find a basis for the vectors in $\R^4$ of the form
\[
  \left(
    \begin{array}{c}
      3a+6b-c \\
      6a-2b-2c \\
      -9a+5b+3c \\
      -3a+b+c
    \end{array}
  \right)
\]
where $a,\,b,\,c \in \R$. \\

\noindent{\bfseries Question 5.} Find a basis for
\[
  H_1 = \{(a,\,b,\,c) : a-3b+c=0,\,b-2c=0,\,2b-c=0\}
\]
and
\[
  H_2 = \{(a,\,b,\,c,\,d) : a-3b+c=0\}.
\] \\

\noindent{\bfseries Question 6.} The the space $C(\R)$ of all continuous functions
on the real line is an infinite dimensional vector space. \\

\noindent{\bfseries Question 7.} For an $n \times n$ matrix $A$, we use
\[
  A\begin{pmatrix}i_1 & \cdots & i_k \\ i_1 & \cdots & i_k\end{pmatrix}
\]
to denote the determinant of the submatrix formed by choosing the rows
$i_1,\dots,\,i_k$ and the columns $j_1,\dots,\,j_k$. Let
$\vartheta_1,\dots,\,\vartheta_n$ be the not necessarily distinct and possibly
complex eigenvalues of $A$. Prove that
\[
  \sum_{1 \leqq i_1 < \cdots < i_k \leqq n} \vartheta_{i_1}\cdots\vartheta_{i_k}
  = \sum_{1 \leqq i_1 < \cdots < i_k \leqq n} A
  \begin{pmatrix}
    i_1 & \cdots & i_k \\ j_1 & \cdots & j_k
  \end{pmatrix}.
\]
Use this to prove $\text{tr}(A)=\sum_{i=1}^n A_{i,i} = \sum_{i=1}^n \vartheta_i$
and $\text{det}(A)=\vartheta_1 \cdots \vartheta_n$. [Hint: You will need to
consider the characteristic equation $\text{det}(xI-A)$ and the multilinearity
of the determinant.]

\end{document}