\documentclass[a4paper,11pt]{article}

\usepackage[utf8]{inputenc}
\usepackage[english]{babel}
\usepackage{amssymb, amsmath, amsthm, mathrsfs}
\usepackage[left=1.0in,right=1.0in,top=1.0in,bottom=1.0in]{geometry}
\usepackage[inline,shortlabels]{enumitem}
\usepackage{times}
\usepackage{xcolor}

\newcommand{\R}{\mathbf{R}}
\newcommand{\C}{\mathbf{C}}
\newcommand{\PP}{\mathbf{P}}
\newcommand{\ddim}{\text{dim}}
\newcommand{\blue}[1]{\textcolor{blue}{#1}}

\begin{document}

\begin{center}
  {\Large\bfseries Math 240 Tutorial \\ Solutions}
\end{center}
\begin{center}
  {\bfseries July 18}
\end{center}

\noindent{\bfseries Question 1.} For the following matrices, give a basis for their
null space.
\begin{enumerate}[(a)]
\item
  \[
    A =
    \left(
      \begin{array}{rrrr}
        1 & 3 & 5 & 0 \\ 0 & 1 & 4 & -2
      \end{array}
    \right).
  \] \\

  \blue{We have
    \[
      A \sim
      \left(
        \begin{array}{rrrr}
          1 & 0 & -7 & 6 \\ 0 & 1 & 4 & -2
        \end{array}
      \right).
    \]
    So every vector in $\text{ker}(A)$ can be written as
    \[
      \left(
        \begin{array}{c}
          7x-6y \\ -4x+2y \\ x \\ y
        \end{array}
      \right)
      =x
      \left(
        \begin{array}{r}
          7 \\ -4 \\ 1 \\ 0
        \end{array}
      \right)
      +y
      \left(
        \begin{array}{r}
          -6 \\ 2 \\ 0 \\ 1
        \end{array}
      \right).
    \]
    Therefore, a basis is given by
    \[
      \left\{
        \left(
          \begin{array}{r}
            7 \\ -4 \\ 1 \\ 0
          \end{array}
        \right),
        \left(
          \begin{array}{r}
            -6 \\ 2 \\ 0 \\ 1
          \end{array}
        \right)
      \right\}.
    \] \\}
  
\item
  \[
    A =
    \left(
      \begin{array}{rrrr}
        1 & -6 & 4 & 0 \\ 0 & 0 & 2 & 0
      \end{array}
    \right).
  \] \\

  \blue{We have
    \[
      A \sim
      \left(
        \begin{array}{rrrr}
          1 & -6 & 0 & 0 \\ 0 & 0 & 1 & 0
        \end{array}
      \right).
    \]
    So every vector in $\text{ker}(A)$ can be written as
    \[
      \left(
        \begin{array}{c}
          6x \\ x \\ 0 \\ y
        \end{array}
      \right)
      =x
      \left(
        \begin{array}{r}
          6 \\ 1 \\ 0 \\ 0
        \end{array}
      \right)
      +y
      \left(
        \begin{array}{r}
          0 \\ 0 \\ 0 \\ 1
        \end{array}
      \right).
    \]
    Therefore, a basis is given by
    \[
      \left\{
        \left(
          \begin{array}{r}
            6 \\ 1 \\ 0 \\ 0
          \end{array}
        \right),
        \left(
          \begin{array}{r}
            0 \\ 0 \\ 0 \\ 1
          \end{array}
        \right)
      \right\}.
    \] \\}
  
\item
  \[
    A =
    \left(
      \begin{array}{rrrrr}
        1 & -2 & 0 & 4 & 0 \\
        0 & 0 & 1 & -9 & 0 \\
        0 & 0 & 0 & 0 & 1
      \end{array}
    \right).
  \] \\

  \blue{The matrix $A$ is alread in reduced row echelon form, so we see that
    every vector in $\text{ker}(A)$ can be written as
    \[
      \left(
        \begin{array}{c}
          2x-4y \\ x \\ 9y \\ y \\ 0
        \end{array}
      \right)
      =x
      \left(
        \begin{array}{r}
          2 \\ 1 \\ 0 \\ 0 \\ 0
        \end{array}
      \right)
      +y
      \left(
        \begin{array}{r}
          -4 \\ 0 \\ 9 \\ 1 \\ 0
        \end{array}
      \right).
    \]
    Therefore, a basis is given by
    \[
      \left\{
        \left(
          \begin{array}{r}
            2 \\ 1 \\ 0 \\ 0 \\ 0
          \end{array}
        \right),
        \left(
          \begin{array}{r}
            -4 \\ 0 \\ 9 \\ 1 \\ 0
          \end{array}
        \right)
      \right\}.
    \] \\}
\end{enumerate}

\noindent{\bfseries Question 2.} Find a basis for the space spanned by
\[
  \left(
    \begin{array}{r}
      -8\\7\\6\\5\\-7
    \end{array}
  \right),
  \left(
    \begin{array}{r}
      8\\-7\\-9\\-5\\7
    \end{array}
  \right),
  \left(
    \begin{array}{r}
      -8\\7\\4\\5\\-7
    \end{array}
  \right),
  \left(
    \begin{array}{r}
      1\\4\\9\\6\\-7
    \end{array}
  \right),
  \left(
    \begin{array}{r}
      -9\\3\\-4\\-1\\0
    \end{array}
  \right).
\] \\

\blue{We reduce the follow matrix
  \begin{align*}
    \left(
      \begin{array}{rrrrr}
        -8 & 8 & -8 & 1 & -9 \\
        7 & -7 & 7 & 4 & 3 \\
        6 & -9 & 4 & 9 & -4 \\
        5 & -5 & 5 & 6 & -1 \\
        -7 & 7 & -7 & -7 & 0
      \end{array}
    \right)
    &\sim
    \left(
      \begin{array}{rrrrr}
        1 & 0 & \frac{5}{3} & 0 & \frac{4}{3} \\
        0 & 1 & \frac{2}{3} & 0 & \frac{1}{3} \\
        0 & 0 & 0 & 1 & -1 \\
        0 & 0 & 0 & 0 & 0 \\
        0 & 0 & 0 & 0 & 0
      \end{array}
    \right).
  \end{align*}
  So the span of the vectors is the column space of the matrix which has the
  basis
  \[
    \left\{
      \left(
        \begin{array}{r}
          -8\\7\\6\\5\\-7
        \end{array}
      \right),
      \left(
        \begin{array}{r}
          8\\-7\\-9\\-5\\7
        \end{array}
      \right),
      \left(
        \begin{array}{r}
          1\\4\\9\\6\\-7
        \end{array}
      \right)
    \right\}.
  \] \\}

\noindent{\bfseries Question 3.} Given vectors $\vec u_1, \dots,\, \vec u_p$ in a
vector space $V$, show $\vec x$ is a linear combination of $\vec u_1, \dots,\,
\vec u_p$ if and only if $[\vec x]_B$ is a linear combination of $[\vec u_1]_B,
\dots,\, [\vec u_p]_B$. \\

\blue{This follows at once because change of basis is an invertible linear map.
  \\}

\noindent{\bfseries Question 4.} Find a basis for the vectors in $\R^4$ of the form
\[
  \left(
    \begin{array}{c}
      3a+6b-c \\
      6a-2b-2c \\
      -9a+5b+3c \\
      -3a+b+c
    \end{array}
  \right)
\]
where $a,\,b,\,c \in \R$. \\

\blue{We can write the vectors as
  \[
    a+
    \left(
      \begin{array}{r}
        3 \\ 6 \\ -9 \\ -3
      \end{array}
    \right)
    +b
    \left(
      \begin{array}{r}
        6 \\ -2 \\ 5 \\ 1
      \end{array}
    \right)
    +c
    \left(
      \begin{array}{r}
        -1 \\ -2 \\ 3 \\ 1
      \end{array}
    \right)
  \]
  A maximal linearly independent set of these three vectors is given by
  \[
    \left\{
      \left(
        \begin{array}{r}
          3 \\ 6 \\ -9 \\ -3
        \end{array}
      \right),
      \left(
        \begin{array}{r}
          6 \\ -2 \\ 5 \\ 1
        \end{array}
      \right)
    \right\}.
  \] \\}

\noindent{\bfseries Question 5.} Find a basis for
\[
  H_1 = \{(a,\,b,\,c) : a-3b+c=0,\,b-2c=0,\,2b-c=0\}
\]
and
\[
  H_2 = \{(a,\,b,\,c,\,d) : a-3b+c=0\}.
\] \\

\blue{We recognize $H_1$ has the kernel of the matrix
  \[
    A_1=
    \left(
      \begin{array}{rrr}
        1 & -3 & 1 \\ 0 & 1 & -2 \\ 0 & 2 & -1
      \end{array}
    \right).
  \]
  $A_1$ is invertible, so $\ker(A_1)=\{\vec 0\}$ and $\text{dim ker}(A_1)=0$.}

\blue{$H_2$ is the kernel of a linear functional and has dimension $4-1=3$. It
  is the kernel of the matrix
  \[
    A_2 = \begin{pmatrix}
            1 & -3 & 1 & 0
          \end{pmatrix}.
  \]
  The vectors of this space are given by
  \[
    \left(
      \begin{array}{c}
        3b+c \\ b \\ c \\ d
      \end{array}
    \right)
    =b
    \left(
      \begin{array}{r}
        3 \\ 1 \\ 0 \\ 0
      \end{array}
    \right)
    +c
    \left(
      \begin{array}{r}
        1 \\ 0 \\ 1 \\ 0
      \end{array}
    \right)
    +d
    \left(
      \begin{array}{r}
        0 \\ 0 \\ 0 \\ 1
      \end{array}
    \right).
  \] \\}

\noindent{\bfseries Question 6.} The the space $C(\R)$ of all continuous functions
on the real line is an infinite dimensional vector space. \\

\blue{We already have that the set $C(\R)$ is a vector space under componentwise
addition. That it is infinite dimensional follows from the fact that the space
of all polynomials of finite degree $\PP_\infty$ is a proper subspace. \\}

\noindent{\bfseries Question 7.} For an $n \times n$ matrix $A$, we use
\[
  A\begin{pmatrix}i_1 & \cdots & i_k \\ j_1 & \cdots & j_k\end{pmatrix}
\]
to denote the determinant of the submatrix formed by choosing the rows
$i_1,\dots,\,i_k$ and the columns $j_1,\dots,\,j_k$. Let
$\vartheta_1,\dots,\,\vartheta_n$ be the not necessarily distinct and possibly
complex eigenvalues of $A$. Prove that
\[
  \sum_{1 \leqq i_1 < \cdots < i_k \leqq n} \vartheta_{i_1}\cdots\vartheta_{i_k}
  = \sum_{1 \leqq i_1 < \cdots < i_k \leqq n} A
  \begin{pmatrix}
    i_1 & \cdots & i_k \\ i_1 & \cdots & i_k
  \end{pmatrix}.
\]
Use this to prove $\text{tr}(A)=\sum_{i=1}^n A_{i,i} = \sum_{i=1}^n \vartheta_i$
and $\text{det}(A)=\vartheta_1 \cdots \vartheta_n$. [Hint: You will need to
consider the characteristic equation $\text{det}(xI-A)$ and the multilinearity
of the determinant.] \\

% \blue{By induction on $n$. The base case $n=1$ is immediate. Assume the result
%   holds for $n-1$. We show it must hold for $n$ as well. Define
%   \[
%     E_k = \sum_{1 \leqq i_1 < \cdots < i_k \leqq n} A
%     \begin{pmatrix}
%       i_1 & \cdots & i_k \\ i_1 & \cdots & i_k
%     \end{pmatrix}.
%   \]
%   Observe
%   \begin{align*}
%     \begin{pmatrix}
%       A_{1,1}-x & A_{1,2} & \cdots & A_{1,n} \\
%       A_{2,1} & A_{2,2}-x & \cdots & A_{2,n} \\
%       \vdots & \vdots & \ddots & \vdots \\
%       A_{n,1} & A_{n,2} & \cdots & A_{n,n}-x
%     \end{pmatrix}
%     &= \text{det}(A)
%       \begin{pmatrix}
%         -x & A_{1,2} & \cdots & A_{1,n} \\
%         0 & A_{2,2}-x & \cdots & A_{2,n} \\
%         \vdots & \vdots & \ddots & \vdots \\
        
%       \end{pmatrix}
%   \end{align*}
% }

\end{document}