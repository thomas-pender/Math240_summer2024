\documentclass[a4paper,11pt]{article}

\usepackage[utf8]{inputenc}
\usepackage[english]{babel}
\usepackage{amssymb, amsmath, amsthm, mathrsfs}
\usepackage[left=1.0in,right=1.0in,top=1.0in,bottom=1.0in]{geometry}
\usepackage[inline,shortlabels]{enumitem}
\usepackage{times}
\usepackage{xcolor}

\newcommand{\R}{\mathbf{R}}
\newcommand{\C}{\mathbf{C}}
\newcommand{\PP}{\mathbf{P}}
\newcommand{\ddim}{\text{dim}}
\newcommand{\BB}[1]{\textcolor{blue}{#1}}

\begin{document}

\begin{center}
  {\Large\bfseries Math 240 Tutorial \\ Solutions}
\end{center}
\begin{center}
  {\bfseries July 11}
\end{center}

\noindent{\bfseries Question 1.} Consider the matrix
\[
  A =
  \left(
    \begin{array}{rrrr}
      1 & -1 & -1 & -1 \\
      -1 & 1 & -1 & -1 \\
      -1 & -1 & 1 & -1 \\
      -1 & -1 & -1 & 1 \\
    \end{array}
  \right).
\]
\begin{enumerate}[(a)]
\item Calculate the determinant of $A$ using
  \begin{enumerate*}[(1)]
  \item cofactor expansion, and
  \item row reduction. \\
  \end{enumerate*}

  \BB{Using cofactor expansion along the first row, we have
    \begin{align*}
      \text{det}(A) &=
      \left|
      \begin{array}{rrr}
        1 & -1 & -1 \\ -1 & 1 & -1 \\ -1 & -1 & 1
      \end{array}
      \right| + 
      \left|
      \begin{array}{rrr}
        -1 & -1 & -1 \\ -1 & 1 & -1 \\ -1 & -1 & 1
      \end{array}
      \right| -
      \left|
      \begin{array}{rrr}
        -1 & 1 & -1 \\ -1 & -1 & -1 \\ -1 & -1 & 1
      \end{array}
      \right| +
      \left|
      \begin{array}{rrr}
        -1 & 1 & -1 \\ -1 & -1 & 1 \\ -1 & -1 & -1
      \end{array}
      \right| \\
      &=
      \left|
      \begin{array}{rrr}
        1 & -1 & -1 \\ -1 & 1 & -1 \\ -1 & -1 & 1
      \end{array}
      \right| + 3
      \left|
      \begin{array}{rrr}
        -1 & -1 & -1 \\ -1 & 1 & -1 \\ -1 & -1 & -1
      \end{array}
      \right| \\
      &=
      8\left|
      \begin{array}{rr}
        -1 & -1 \\ -1 & 1
      \end{array}
      \right| \\
          &= 8(-2) \\
          &= -16.
    \end{align*}
    Using row reduction,
    \begin{align*}
      \text{det}(A) &=
      \left|
      \begin{array}{rrrr}
        1 & -1 & -1 & -1 \\
        -1 & 1 & -1 & -1 \\
        -1 & -1 & 1 & -1 \\
        -1 & -1 & -1 & 1 \\
      \end{array}
      \right|                     
      =
      \left|
      \begin{array}{rrrr}
        1 & -1 & -1 & -1 \\
        0 & 0 & -2 & -2 \\
        0 & -2 & 0 & -2 \\
        0 & -2 & -2 & 0
      \end{array}
      \right|
      =
      \left|
      \begin{array}{rrrr}
        1 & -1 & -1 & -1 \\
        0 & 0 & -2 & -2 \\
        0 & -2 & 0 & -2 \\
        0 & 0 & -2 & 2
      \end{array}
      \right| \\
      &=
      \left|
      \begin{array}{rrrr}
        1 & -1 & -1 & -1 \\
        0 & 0 & -2 & -2 \\
        0 & -2 & 0 & -2 \\         
        0 & 0 & 0 & 4
      \end{array}
      \right|
      = (-1)
      \left|
      \begin{array}{rrrr}
        1 & -1 & -1 & -1 \\
        0 & -2 & 0 & -2 \\
        0 & 0 & -2 & -2 \\
        0 & 0 & 0 & 4
      \end{array}
      \right| \\
      &= (-1)(-2)(-2)(4) \\
      &= -16.
    \end{align*}
  }
  
\item Calculate the inverse using
  \begin{enumerate*}[(1)]
  \item the adjugate of $A$, and
  \item row reduction. \\
  \end{enumerate*}

  \BB{Omitting the tedious details, we have
    \[
      \text{adj}(A) =
      \left(
        \begin{array}{rrrr}
          -4 & 4 & 4 & 4 \\
          4 & -4 & 4 & 4 \\
          4 & 4 & -4 & 4 \\
          4 & 4 & 4 & -4 \\
        \end{array}
      \right)
    \]
    so that
    \begin{align*}
      A^{-1} &= \frac{1}{\text{det}(A)}\text{adj}(A) \\
      &= \frac{1}{4}
        \left(
        \begin{array}{rrrr}
          1 & -1 & -1 & -1 \\
          -1 & 1 & -1 & -1 \\
          -1 & -1 & 1 & -1 \\
          -1 & -1 & -1 & 1 \\
        \end{array}
      \right).
    \end{align*}
  }
  
\item Do the columns (rows) of $A$ form a basis for $\R^4$? If they do, give
  the change of basis matrix from the standard basis of $\R^4$ to the columns of
  $A$. \\

  \BB{Since the matrix is invertible, the columns of $A$ form a basis for
    $\R^4$. The change of basis matrix is simply $A^{-1}$. \\}
\end{enumerate}

\noindent{\bfseries Question 2.} Consider the matrices
\[
  A =
  \left(
    \begin{array}{rrr}
      -1 & 3 & -1 \\
      -3 & 5 & -1 \\
      -3 & 3 & 1
    \end{array}
  \right), \qquad
  P = 
  \left(
    \begin{array}{rrr}
      1 & 1 & -1 \\
      1 & 1 & 0 \\
      1 & 0 & 3
    \end{array}
  \right).
\]
The matrix $P$ is invertible. Find $P^{-1}$ by any means and calculate
$P^{-1}AP=D$. Prove that $A$ is invertible if and only if $D$ is invertible. If
$A$ is invertible, find its inverse by first finding the inverse of $D$ and then
multiplying $D^{-1}$ by $P^{-1}$ and $P$ in some order. \\ 

\BB{We have
  \[
    P^{-1} =
    \left(
      \begin{array}{rrr}
        3 & -3 & 1 \\
        -3 & 4 & -1 \\
        -1 & 1 & 0
      \end{array}
    \right)
  \]
  so that
  \[
    P^{-1}AP = D =
    \left(
      \begin{array}{rrr}
        1 & 0 & 0 \\
        0 & 2 & 0 \\
        0 & 0 & 2
      \end{array}
    \right).
  \]
  Since $D$ is a diagonal matrix with each diagonal entry nonzero, it follows
  that it is invertible with inverse
  \[
    D^{-1} =
    \left(
      \begin{array}{rrr}
        1 & 0 & 0 \\
        0 & 1/2 & 0 \\
        0 & 0 & 1/2
      \end{array}
    \right).
  \]
  From the properties of matrix inversion, we have
  \[
    P^{-1}A^{-1}P = D^{-1}
  \]
  so that
  \[
    P^{-1}D^{-1}P = A^{-1}.
  \]
  Doing the calculation, we find
  \[
    A^{-1} =
    \left(
      \begin{array}{rrr}
        2 & -3/2 & 1/2 \\
        3/2 & -1 & 1/2 \\
        3/2 & -3/2 & 1
      \end{array}
    \right).
  \] \\
}

\noindent{\bfseries Question 3.} Prove that the linear transformations of $\R^2$
consisting of compositions of reflections and rotations have determinants $\pm
1$. \\

\BB{Using the geometry of $\R^2$, we see that every such transformation can be
  written as a product
  \[
    \left(
      \begin{array}{rr}
        -1 & 0 \\ 0 & 1
      \end{array}
    \right)
    \left(
      \begin{array}{rr}
        \cos\vartheta & -\sin\vartheta \\ \sin\vartheta & \cos\vartheta
      \end{array}
    \right).
  \]
  The determinant of the right factor is 1, while the determinant of the left
  factor is $-1$. \\}

\noindent{\bfseries Question 4.} An isomorphism is an invertible linear
transformation from one vector space onto another. Give two distinct
isomorphisms $\PP_3 \rightarrow \R^3$. NB: $\R^n$ (or $\C^n$) are often referred
to as the coordinate spaces. This question shows that the coordinate
representation of a vector is not in general unique; it depends on the choice of
basis. \\

\BB{One isomorphism is the standard one given by $e_i \leftrightarrows x^i$. For
another isomorphism, we simply need another basis for $\PP_3$. One such basis is
given by $\{1, x, x(x-1), x(x-1)(x-2)\}$. The isomorphism isn't difficult to
find in this case as well. \\}

\noindent{\bfseries Question 5.} Define
\begin{align*}
  H &=
      \left\{
      \begin{pmatrix}
        u & -u-x \\ 0 & x
      \end{pmatrix}
                        : u,x \in \R
                        \right\}, \\
  K &=
      \left\{
      \begin{pmatrix}
        v & 0 \\ w & -v
      \end{pmatrix}
                     \right\}.
\end{align*}
Do the following.
\begin{enumerate}[(a)]
\item $H$ and $K$ are subspaces of $M_{2 \times 2}(\R)$.

  \BB{Note that $O \in H$ so that $H \neq \emptyset$. Define
    \[
      A_1 = \begin{pmatrix}
              u_1 & -u_1-x_1 \\
              0 & x_1
            \end{pmatrix}, \qquad
            A_2 = \begin{pmatrix}
                    u_2 & -u_2-x_2 \\
                    0 & x_2
                  \end{pmatrix}.
                \]
                Then
                \[
                  A_1+A_2 =
                  \begin{pmatrix}
                    u_1+u_2 & -(u_1+u_2)-(x_1+x_2) \\
                    0 & x_1+x_2
                  \end{pmatrix}
                  \in H,
                \]
                and
                \[
                  \alpha A_1 = \begin{pmatrix}
                                 \alpha u_1 & -\alpha u_1-\alpha x_1 \\
                                 0 & \alpha x_1
                               \end{pmatrix}
                               \in H.
                             \]
                             Hence, $H$ is a subspace of $M_{2 \times 2}(\R)$, Similarly, $K$ is also a
                             subspace of $M_{2 \times 2}(\R)$. \\}

\item Construct bases for $H$, $K$, $H+K$, and $H \cap K$. \\

  \BB{Let $E_{i,j}$ be as in the solution to Question 3. Then
    $\{E_{1,1}-E_{1,2},E_{2,2}-E_{1,2}\}$ is a basis for $H$,
    $\{E_{1,1}-E_{2,2},E_{2,1}\}$ is a basis for $K$, and $\{E_{1,1}-E_{2,2}\}$
    is a basis for $H \cap K$. Finally, we note that
    \[
      H+K=\text{span}\{E_{1,1}-E_{1,2},E_{2,2}-E_{1,2},E_{1,1}-E_{2,2},E_{2,1}\}
      = \text{span}\{E_{1,1}-E_{1,2},E_{2,2}-E_{1,2},E_{2,1}\}.
    \]
    Since $\{E_{1,1}-E_{1,2},E_{2,2}-E_{1,2},E_{2,1}\}$ is linearly independent,
    this is a basis for $H+K$. \\}
\end{enumerate}

\noindent{\bfseries Question 6.} Answer whether the following are subspaces of
$\R^3$. The set of points $(x,\,y,\,z) \in \R^3$ such that
\begin{enumerate}[(a)]
\item $x+2y-3z=4$, \\

  \BB{No. \\}

\item $\frac{x-1}{2}=\frac{y+2}{3}=\frac{z}{4}$, \\

  \BB{No. \\}
  
\item $x+y+z=0$ and $x-y+z=1$, \\

  \BB{No. \\}
  
\item $x=-z$ and $x=z$, \\

  \BB{Yes. \\}

\item $x^2+y^2=z$, or \\

  \BB{No. \\}
  
\item $\frac{x}{2} = \frac{y-3}{5}$. \\

  \BB{No. \\}
\end{enumerate}

\end{document}