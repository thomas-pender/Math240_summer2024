\documentclass[a4paper,11pt]{article}

\usepackage[utf8]{inputenc}
\usepackage[english]{babel}
\usepackage{amssymb, amsmath, amsthm, mathrsfs}
\usepackage[left=1.0in,right=1.0in,top=1.0in,bottom=1.0in]{geometry}
\usepackage[inline,shortlabels]{enumitem}
\usepackage{times}
\usepackage{xcolor}

\newcommand{\R}{\mathbf{R}}
\newcommand{\C}{\mathbf{C}}
\newcommand{\PP}{\mathbf{P}}
\newcommand{\ddim}{\text{dim}}

\begin{document}

\begin{center}
  {\Large\bfseries Math 240 Tutorial \\ Questions}
\end{center}
\begin{center}
  {\bfseries July 11}
\end{center}

\noindent{\bfseries Question 1.} Consider the matrix
\[
  A =
  \left(
    \begin{array}{rrrr}
      1 & -1 & -1 & -1 \\
      -1 & 1 & -1 & -1 \\
      -1 & -1 & 1 & -1 \\
      -1 & -1 & -1 & 1 \\
    \end{array}
  \right).
\]
\begin{enumerate}[(a)]
\item Calculate the determinant of $A$ using
  \begin{enumerate*}[(1)]
  \item cofactor expansion, and
  \item row reduction.
  \end{enumerate*}
\item Calculate the inverse using
  \begin{enumerate*}[(1)]
  \item the adjugate of $A$, and
  \item row reduction.
  \end{enumerate*}
\item Do the columns (rows) of $A$ form a basis for $\R^4$? If they do, give
  the change of basis matrix from the standard basis of $\R^4$ to the columns of
  $A$. \\
\end{enumerate}

\noindent{\bfseries Question 2.} Consider the matrices
\[
  A =
  \left(
    \begin{array}{rrr}
      -1 & 3 & -1 \\
      -3 & 5 & -1 \\
      -3 & 3 & 1
    \end{array}
  \right), \qquad
  P = 
  \left(
    \begin{array}{rrr}
      1 & 1 & -1 \\
      1 & 1 & 0 \\
      1 & 0 & 3
    \end{array}
  \right).
\]
The matrix $P$ is invertible. Find $P^{-1}$ by any means and calculate
$P^{-1}AP=D$. Prove that $A$ is invertible if and only if $D$ is invertible. If
$A$ is invertible, find its inverse by first finding the inverse of $D$ and then
multiplying $D^{-1}$ by $P^{-1}$ and $P$ in some order. \\ 

\noindent{\bfseries Question 3.} Prove that the linear transformations of $\R^2$
consisting of compositions of reflections and rotations have determinants $\pm
1$. \\

\noindent{\bfseries Question 4.} An isomorphism is an invertible linear
transformation from one vector space onto another. Give two distinct
isomorphisms $\PP_3 \rightarrow \R^3$. NB: $\R^n$ (or $\C^n$) are often referred
to as the coordinate spaces. This question shows that the coordinate
representation of a vector is not in general unique; it depends on the choice of
basis. \\

\noindent{\bfseries Question 5.} Define
\begin{align*}
  H &=
      \left\{
      \begin{pmatrix}
        u & -u-x \\ 0 & x
      \end{pmatrix}
                        : u,x \in \R
                        \right\}, \\
  K &=
      \left\{
      \begin{pmatrix}
        v & 0 \\ w & -v
      \end{pmatrix}
                     \right\}.
\end{align*}
Do the following.
\begin{enumerate}[(a)]
\item $H$ and $K$ are subspaces of $M_{2 \times 2}(\R)$.
\item Construct bases for $H$, $K$, $H+K$, and $H \cap K$. \\
\end{enumerate}

\noindent{\bfseries Question 6.} Answer whether the following are subspaces of
$\R^3$. The set of points $(x,\,y,\,z) \in \R^3$ such that
\begin{enumerate}[(a)]
\item $x+2y-3z=4$,
\item $\frac{x-1}{2}=\frac{y+2}{3}=\frac{z}{4}$,
\item $x+y+z=0$ and $x-y+z=1$,
\item $x=-z$ and $x=z$,
\item $x^2+y^2=z$, or
\item $\frac{x}{2} = \frac{y-3}{5}$.
\end{enumerate}

\end{document}