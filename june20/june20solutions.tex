\documentclass[a4paper,11pt]{article}

\usepackage[utf8]{inputenc}
\usepackage[english]{babel}
\usepackage{amssymb, amsmath, amsthm, mathrsfs}
\usepackage[left=1.0in,right=1.0in,top=1.0in,bottom=1.0in]{geometry}
\usepackage[inline,shortlabels]{enumitem}
\usepackage{times}
\usepackage{xcolor}

\newcommand{\R}{\mathbf{R}}
\newcommand{\PP}{\mathbf{P}}
\newcommand{\BB}[1]{\textcolor{blue}{#1}}

\begin{document}

\begin{center}
  {\Large\bfseries Math 240 Tutorial \\ Solutions}
\end{center}
\begin{center}
  {\bfseries June 20}
\end{center}

\noindent{\bfseries Question 1.} Show the following for $\R^n$.
\begin{enumerate}[(a)]
\item Show that scalar multiplication is a linear transformation. \\

  \BB{Fix $a \in \R$, and let $T:\R^n \rightarrow \R^n$ be the map $T(\vec
    v)=a\vec v$. Then $T(b\vec v)=ab\vec v=ba\vec v=bT(\vec v)$ and $T(\vec
    v+\vec u)=a(\vec v+\vec u)=a\vec v+a\vec u=T(\vec v)+T(\vec u)$. We have
    shown that $T$ is linear. \\}

\item When is this linear map invertible? \\

  \BB{This map is invertible precisely in the case the scalar by which we are
    multiplying is nonzero. \\}
  
\item Is its inverse a linear transformation? \\

  \BB{Let $T$ be as in part (a), and assume that $a \neq 0$. Then $T$ is
    invertible and $T^{-1}$ is given by multiplcation by $a^{-1}$. Since this
    is multiplcation by a scalar, it is linear. \\}
  
\item Fix an element $a \in \R^n$. What is the matrix corresponding to the
linear transformation $\vec v \mapsto a\vec v$ with respect to the standard
spanning vectors? \\

\BB{Recall the standard spanning vectors are $\{\vec e_1,\,\vec e_2,\dots,\,\vec
  e_n\}$ where $\vec e_i$ is the vector with a 1 in position $i$ and zeros
  everywhere else. Then the matrix corresponding to multiplication by $a$ is
  givn by
  \[
    [T(\vec e_1) | T(\vec e_2) | \cdots | T(\vec e_n)]=
    [a\vec e_1 | a\vec e_2 | \cdots | a\vec e_n] = aI.
  \]
}
\end{enumerate}

\noindent{\bfseries Question 2.} Construct the standard matrix for the
transformation that rotates the vectors of $\R^2$ by $-\pi/6$ radians. \\

\BB{Note that
  \begin{align*}
    e_1&=(1,0) \mapsto (\cos(-\pi/6),\sin(-\pi/6))=(\sqrt{3}/2,-1/2), \\
    e_2&=(0,1) \mapsto (\cos(\pi/3),\sin(\pi/3))=(1/2,\sqrt{3}/2).
    \end{align*}
    It follows that the standard matrix is given by
    \[
      \left(
        \begin{array}{rr}
          \frac{\sqrt{3}}{2} & \frac{1}{2} \\
          -\frac{1}{2} & \frac{\sqrt{3}}{2}
        \end{array}
      \right).
    \] \\
}

\noindent{\bfseries Question 3.} Define $T : \R^3 \rightarrow \R^4$ by
\[
  T(\vec x) = (x_1-x_3,\, x_1+x_2,\,x_3-x_2,\,x_1-2x_2).
\]
\begin{enumerate}[(a)]
\item Is $T$ linear? \\

  \BB{It is easily verified that $T(\vec x_1 + \vec x_2)=T(\vec x_1) + T(\vec
    x_2)$ and $T(\alpha\vec x)=\alpha T(\vec x)$ for every $\alpha \in \R$ and
    $x_1,\,\vec x_2 \in \R^3$. \\}
  
\item What is $T(1,-2,3)$? \\

  \BB{By definition, $T(1,-2,3)=(-2,-1,5,5)$. \\}
  
\item Find a vector $\vec x \in \R^3$ such that $T(\vec x)=(8,9,-5,0)$. \\

  \BB{We solve the system
    \begin{align*}
      x_1 - x_3 &= 8, \\
      x_1 + x_2 &= 9, \\
      x_3 - x_2 &= -5, \\
      x_1 - 2x_2 &= 0
    \end{align*}
    to find the unique preimage is $(6,3,-2)$. \\}
  
\item What is the standard basis for $T$? \\

  \BB{Observe
    \begin{align*}
      e_1 &\mapsto (1,1,0,1), \\
      e_2 &\mapsto (0,1,-1,-2), \\
      e_3 &\mapsto (-1,0,1,0)
    \end{align*}
    so the standard matrix is
    \[
      \left(
        \begin{array}{rrr}
          1 & 0 & -1 \\
          1 & 1 & 0 \\
          0 & -1 & 1 \\
          1 & -2 & 0
        \end{array}
      \right).
    \] \\
  }
\end{enumerate}

\noindent{\bfseries Question 4.} Calculate the following determinants.
\begin{enumerate}[(a)]
\item
  \[
    \text{det}
    \left(
      \begin{array}{rrrr}
        6 & 9 & 39 & 49 \\
        5 & 7 & 32 & 37 \\
        3 & 4 & 4 & 4 \\
        1 & 1 & 1 & 1
      \end{array}
    \right).
  \] \\

  \BB{Note
    \[
      \left(
        \begin{array}{rrrr}
          6 & 9 & 39 & 49 \\
          5 & 7 & 32 & 37 \\
          3 & 4 & 4 & 4 \\
          1 & 1 & 1 & 1
        \end{array}
      \right)
      \sim
      \left(
        \begin{array}{rrrr}
          1 & 1 & 1 & 1 \\
          0 & 1 & 1 & 1 \\
          0 & 0 & 30 & 40 \\
          0 & 0 & 0 & -\frac{10}{3}
        \end{array}
      \right)
    \]
    where we have made 2 row swaps. So the determinant is
    $30(-\frac{10}{3})=-100$. \\}
  
\item
  \[
    \text{det}
    \left(
      \begin{array}{rrrr}
        1 & 0 & 1 & 1 \\
        1 & -1 & 2 & 0 \\
        2 & -1 & 3 & 1 \\
        4 & 17 & 0 & -5
      \end{array}
    \right).
  \] \\

  \BB{Note
    \[
      \left(
        \begin{array}{rrrr}
          1 & 0 & 1 & 1 \\
          1 & -1 & 2 & 0 \\
          2 & -1 & 3 & 1 \\
          4 & 17 & 0 & -5
        \end{array}
      \right)
      \sim
      \left(
        \begin{array}{rrrr}
          1 & 0 & 1 & 1 \\
          0 & -1 & 1 & -1 \\
          0 & -1 & 1 & -1 \\
          4 & 17 & 0 & -5
        \end{array}
      \right).
    \]
    So the determinant is 0 (there are two equal rows in a row equivalent
    matrix). \\}
  
\item
  \[
    \text{det}
    \left(
      \begin{array}{rrrr}
        13 & 3 & -8 & 6 \\
        0 & 0 & -4 & 0 \\
        1 & 0 & 7 & -2 \\
        3 & 0 & 2 & 0
      \end{array}
    \right).
  \] \\

  \BB{Note
    \[
      \left(
        \begin{array}{rrrr}
          13 & 3 & -8 & 6 \\
          0 & 0 & -4 & 0 \\
          1 & 0 & 7 & -2 \\
          3 & 0 & 2 & 0
        \end{array}
      \right)
      \sim
      \left(
        \begin{array}{rrrr}
          1 & 0 & -2 & 7 \\
          0 & 3 & 32 & -99 \\
          0 & 0 & 6 & -19 \\
          0 & 0 & 0 & -4
        \end{array}
      \right)     
    \]
    where we have had to do 3 row swaps and 1 column swap. Therefore, the
    determinant is given by $(3)(6)(-4)=-72$. \\}
\end{enumerate}

\noindent{\bfseries Question 5.} Solve the following equation for $x$.
\[
  \text{det}
  \left(
    \begin{array}{rrrrrr}
      3 & -4 & 7 & 0 & 6 & -2 \\
      2 & 0 & 1 & 8 & 0 & 0 \\
      3 & 4 & -8 & 3 & 1 & 2 \\
      27 & 6 & 5 & 0 & 0 & 3 \\
      3 & x & 0 & 2 & 1 & -1 \\
      1 & 0 & -1 & 3 & 4 & 0
    \end{array}
  \right)=0.
\] \\

\BB{Evaluating the determinant, we have $3685x+7370=0$ so that $x=-2$. \\}

\noindent{\bfseries Question 6.} Let $M$ be the matrix
\[
  \left(
    \begin{array}{rrrr}
      5 & 4 & -2 & 3 \\
      5 & 7 & -1 & 8 \\
      5 & 7 & 6 & 10 \\
      5 & 7 & 1 & 9
    \end{array}
  \right).
\]
The following hold.
\begin{enumerate}[(a)]
\item $\text{det }M$ can be expressed as the constant 5 times the determinant of
  \[
    \left(
      \begin{array}{rrr}
        3 & 1 & 5 \\
        3 & & \\
        3 & & 
      \end{array}
    \right).
  \]
\item The determinant of the $3 \times 3$ is part (a) can be expressed as the
  constant 3 times the determinant of
  \[
    \left(
      \begin{array}{rr}
        7 & 2 \\
        2 &
      \end{array}
    \right).
  \]
\end{enumerate}
The determinant of the $2 \times 2$ matrix in part (b) is what? Thus the
detmerinant of $M$ is what? \\

\BB{The $2 \times 2$ matrix is
  $\left( \begin{smallmatrix}7&2\\2&1\end{smallmatrix} \right)$. The determinant
of the $2 \times 2$ matrix is $7-4=3$, so the determinant of $M$ is
$(5)(3)(3)=45$. \\}

\noindent{\bfseries Question 7.} Consider again the vector space $\PP_3$, and let
$Q \subseteqq \PP_3$ be the subset of polynomials of degree at most 3 that
vanish when $x=3$. Is $Q$ a subspace? If it is, give a spanning set of $Q$. \\

\BB{Let $p(x) \in \PP_3$ have 3 as a root. Then $p(x)=(x-3)q(x)$ for some
  polynomial $q$ of degree at most 2. So, the subspace under consideration has
  dimension at most 3. In fact, $\{x-3,(x-3)^2,(x-3)^3\}$ are linearly
  independent, so their span is $Q$.}

\end{document}