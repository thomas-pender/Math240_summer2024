\documentclass[a4paper,11pt]{article}

\usepackage[utf8]{inputenc}
\usepackage[english]{babel}
\usepackage{amssymb, amsmath, amsthm, mathrsfs}
\usepackage[left=1.0in,right=1.0in,top=1.0in,bottom=1.0in]{geometry}
\usepackage[inline,shortlabels]{enumitem}
\usepackage{times}
\usepackage{xcolor}

\newcommand{\R}{\mathbf{R}}
\newcommand{\PP}{\mathbf{P}}

\begin{document}

\begin{center}
  {\Large\bfseries Math 240 Tutorial \\ Questions}
\end{center}
\begin{center}
  {\bfseries June 20}
\end{center}

\noindent{\bfseries Question 1.} Show the following for $\R^n$.
\begin{enumerate}[(a)]
\item Show that scalar multiplication is a linear transformation.
\item When is this linear map invertible?
\item Is its inverse a linear transformation?
\item Fix an element $a \in \R^n$. What is the matrix corresponding to the
  linear transformation $\vec v \mapsto a\vec v$? \\
\end{enumerate}

\noindent{\bfseries Question 2.} Construct the standard matrix for the
transformation that rotates the vectors of $\R^2$ by $-\pi/6$ radians. \\

\noindent{\bfseries Question 3.} Define $T : \R^3 \rightarrow \R^4$ by
\[
  T(\vec x) = (x_1-x_3,\, x_1+x_2,\,x_3-x_2,\,x_1-2x_2).
\]
\begin{enumerate}[(a)]
\item Is $T$ linear?
\item What is $T(1,-2,3)$?
\item Find a vector $\vec x \in \R^3$ such that $T(\vec x)=(8,9,-,10)$.
\item What is th standard basis for $T$? \\
\end{enumerate}

\noindent{\bfseries Question 4.} Calculate the following determinants.
\begin{enumerate}[(a)]
\item
  \[
    \text{det}
    \left(
      \begin{array}{rrrr}
        6 & 9 & 39 & 49 \\
        5 & 7 & 32 & 37 \\
        3 & 4 & 4 & 4 \\
        1 & 1 & 1 & 1
      \end{array}
    \right).
  \]
\item
  \[
    \text{det}
    \left(
      \begin{array}{rrrr}
        1 & 0 & 1 & 1 \\
        1 & -1 & 2 & 0 \\
        2 & -1 & 3 & 1 \\
        4 & 17 & 0 & -5
      \end{array}
    \right).
  \]
\item
  \[
    \text{det}
    \left(
      \begin{array}{rrrr}
        13 & 3 & -8 & 6 \\
        0 & 0 & -4 & 0 \\
        1 & 0 & 7 & -2 \\
        3 & 0 & 2 & 0
      \end{array}
    \right).
  \]
\end{enumerate}

\noindent{\bfseries Question 5.} Solve the following equation for $x$.
\[
  \text{det}
  \left(
    \begin{array}{rrrrrr}
      3 & -4 & 7 & 0 & 6 & -2 \\
      2 & 0 & 1 & 8 & 0 & 0 \\
      3 & 4 & -8 & 3 & 1 & 2 \\
      27 & 6 & 5 & 0 & 0 & 3 \\
      3 & x & 0 & 2 & 1 & -1 \\
      1 & 0 & -1 & 3 & 4 & 0
    \end{array}
  \right).
\] \\

\noindent{\bfseries Question 6.} Let $M$ be the matrix
\[
  \left(
    \begin{array}{rrrr}
      5 & 4 & -2 & 3 \\
      5 & 7 & -1 & 8 \\
      5 & 7 & 6 & 10 \\
      5 & 7 & 1 & 9
    \end{array}
  \right).
\]
The following hold.
\begin{enumerate}[(a)]
\item $\text{det }M$ can be expressed as the constant 5 times the determinant of
  \[
    \left(
      \begin{array}{rrr}
        3 & 1 & 5 \\
        3 & & \\
        3 & & 
      \end{array}
    \right).
  \]
\item The determinant of the $3 \times 3$ is part (a) can be expressed as the
  constant 3 times the determinant of
  \[
    \left(
      \begin{array}{rr}
        7 & 2 \\
        2 &
      \end{array}
    \right).
  \]
\end{enumerate}
The determinant of the $2 \times 2$ matrix in part (b) is what? Thus the
detmerinant of $M$ is what? \\

\noindent{\bfseries Question 7.} Consider again the vector space $\PP_3$, and let
$Q \subseteqq \PP_3$ be the subset of polynomials of degree at most 3 that
vanish when $x=3$. Is $Q$ a subspace? If it is, give a spanning set of $Q$. \\

\end{document}